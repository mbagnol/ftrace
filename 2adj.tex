%%% TeX-master: "MAIN" %% Please do not remove this line (hoshino)

\section{\(2\)-adjunction}
\label{2-adjunction}

Let \(\SMCiso\) be the \(2\)-category of symmetric monoidal categories,
strong symmetric monoidal functors and monoidal natural isomorphisms;
and let \(\TSMCiso\) be the \(2\)-category of traced symmetric
monoidal categories, traced symmetric monoidal functors and monoidal
natural isomorphisms. We prove that \(\U\) is a right \(2\)-adjunction.
To see this, we show that
\begin{equation*}
  \TSMCiso(\F\catf{A},\catf{B}) \xrightarrow{\U}
  \TSMCiso(\U\F\catf{A},\U\catf{B}) \xrightarrow{(-) \circ \embed{A}}
  \SMCiso(\catf{A},\U(\catf{\catf{B}})).
\end{equation*}
is an isomorphism between them.

For a symmetric monoidal functor \(\funcf{F} \colon \catf{A} \to
\U(\catf{B})\), we define a traced symmetric monoidal functor
\(\funcf{F}^{\vee} \colon \F\catf{A} \to \catf{B}\) as follows:
\begin{align*}
  \funcf{F}^{\vee}(X) &= \funcf{F}, \\
  \funcf{F}^{\vee}((f,U) \colon X \to Y)
  &= \Tr{\funcf{F}U}_{\funcf{F}X,\funcf{F}Y}((m_{Y,U}^{\funcf{F}})^{-1} \circ \funcf{F}f
  \circ m^{\funcf{F}}_{X,U}), \\
  m^{\funcf{F}^{\vee}}_{\unitobj} &= m^{\funcf{F}}_{\unitobj} \colon \unitobj \to \funcf{F}, \\
  m^{\funcf{F}^{\vee}}_{X,Y} &= m^{\funcf{F}}_{X,Y} \colon \funcf{F}X \otimes \funcf{F}Y
  \to \funcf{F}(X \otimes Y).
\end{align*}
For a monoidal natural isomorphism \(\alpha \colon F \to G\), we
define a monoidal natural isomorphism \(\alpha^{\vee} \colon F \to G\)
to be \(\alpha\). Below, we check that \(\funcf{F}^{\vee}\) is a
traced symmetric monoidal functor and that \(\alpha^{\vee}\) is a
monoidal natural isomorphism. We first check that \(\funcf{F}^{\vee}\)
is a functor. \funcf{F}or \(X \in \catf{A}\),
\begin{align*}
  \funcf{F}^{\vee}(\Id_{X \otimes \unitobj},\unitobj)
  &= \Tr{\funcf{F}\unitobj}_{\funcf{F}X,\funcf{F}X}(\Id_{\funcf{F}X \otimes \funcf{F}\unitobj}) \\
  &= \Tr{\unitobj}_{\funcf{F}X,\funcf{F}X}(\Id_{\funcf{F}X \otimes \unitobj}) \\
  &= \Id_{\funcf{F}X}.
\end{align*}
For \(\F\catf{A}\)-morphisms \((f,U) \colon X \to Y\) and \((g,V)
\colon Y \to Z\), we denote \((m_{Y,U}^{\funcf{F}})^{-1} \circ
\funcf{F}f \circ m^{\funcf{F}}_{X,U}\) by \(f'\) and
\((m_{Z,V}^{\funcf{F}})^{-1} \circ \funcf{F}g \circ
m^{\funcf{F}}_{Y,V}\) by \(g'\). Then
\begin{align*}
  \funcf{F}^{\vee}(g,V) \circ \funcf{F}^{\vee}(f,U)
  &= \Tr{\funcf{F}V}_{\funcf{F}Y,\funcf{F}Z}(g') \circ
  \Tr{\funcf{F}U}_{\funcf{F}X,\funcf{F}Y}(f') \\
  &= \Tr{\funcf{F}U \otimes \funcf{F}V}_{\funcf{F}X,\funcf{F}Z}
  (\funcf{F}Y \otimes \symm{\funcf{F}V}{\funcf{F}U}) \circ
  (g' \otimes \funcf{F}U) \circ (\funcf{F}Y \otimes \symm{\funcf{F}U}{\funcf{F}V}) \circ
  (f' \otimes \funcf{F}V)) \\
  &= \Tr{\funcf{F}U \otimes \funcf{F}V}_{\funcf{F}X,\funcf{F}Z}(((m_{Z,V}^{\funcf{F}})^{-1}
  \otimes \funcf{F}U) \circ
  (m_{Z \otimes V,U}^{\funcf{F}})^{-1} \circ
  \funcf{F}h \circ m_{X \otimes U,V}^{\funcf{F}} \circ (m_{X,U}^{\funcf{F}} \otimes \funcf{F}V)) \\
  &= \Tr{\funcf{F}(U \otimes V)}_{\funcf{F}X,\funcf{F}Z}((m_{Z,U \otimes V}^{\funcf{F}})^{-1} \circ
  \funcf{F}h \circ m_{X, U \otimes V}^{\funcf{F}}) \\
  &= \funcf{F}^{\vee}((g,V) \circ (f,U))
\end{align*}
where \(h \colon X \otimes U \otimes V \to Z \otimes U \otimes V\) is given by
\begin{equation*}
  (Z \otimes \symm{V}{U}) \circ (g \otimes U) \circ (Y \otimes \symm{U}{V})
  \circ (f \otimes V).
\end{equation*}
We next check naturality of \(m^{\funcf{F}^{\vee}}_{X,Y}\). For
\(\F\catf{A}\)-morphisms \((f,U) \colon X \to Z\) and \((g,V) \colon Y
\to W\), we denote \((m^{\funcf{F}}_{Z,U})^{-1} \circ \funcf{F}f \circ m^{\funcf{F}}_{X,U}\) by
\(f'\) and \((m_{W,V}^{\funcf{F}})^{-1} \circ \funcf{F}g \circ m^{\funcf{F}}_{Y,V}\) by \(g'\). Then
\begin{align*}
  & m^{\funcf{F}}_{Z,W} \circ (\funcf{F}^{\vee}(f,U) \otimes \funcf{F}^{\vee}(g,V)) \\
  &= \Tr{\funcf{F}U \otimes \funcf{F}V}_{\funcf{F}X \otimes \funcf{F}Y,\funcf{F}(Z \otimes W)}
  ((m^{\funcf{F}}_{Z,W} \otimes \funcf{F}U \otimes \funcf{F}V) \circ
  (\funcf{F}Z \otimes \symm{\funcf{F}U}{\funcf{F}W} \otimes \funcf{F}V) \circ
  (\funcf{F}f' \otimes \funcf{F}g') \circ
  (\funcf{F}X \otimes \symm{\funcf{F}U}{\funcf{F}Y} \otimes \funcf{F}V)) \\
  &= \Tr{\funcf{F}(U \otimes V)}_{\funcf{F}X \otimes \funcf{F}Y,\funcf{F}(Z \otimes W)}
  ((m_{Z \otimes W, U \otimes V}^{\funcf{F}})^{-1} \circ
  \funcf{F}h \circ
  m_{X \otimes Y, U \otimes V}^{\funcf{F}} \circ
  (m_{X,Y}^{\funcf{F}} \otimes \funcf{F}(U \otimes V))) \\
  &= \funcf{F}^{\vee}((f,U) \otimes (g,V)) \circ m^{\funcf{F}}_{X,Y}
\end{align*}
where \(h \colon X \otimes Y \otimes U \otimes V \to Z \otimes W \otimes U \otimes V\)
is given by
\begin{equation*}
  h = (Z \otimes \symm{U}{W} \otimes V) \circ (f \otimes g) \circ
  (X \otimes \symm{Y}{U} \otimes V).
\end{equation*}
Preservation of trace operators by \(\funcf{F}^{\vee}\) follows from
the associativity axiom. Coherence conditions for
\(m^{\funcf{F}^{\vee}}\) follows from that \(\funcf{F}\) is a symmetric
monoidal functor. Lastly, we check that \(\alpha^{\vee}\) is a
monoidal natural isomorphism. Since coherence conditions for
\(\alpha^{\vee}\) follow from those for \(\alpha\), we only check
naturality of \(\alpha^{\vee}\). For a \(\F\catf{A}\)-morphism \((f,U)
\colon X \to Y\),
\begin{align*}
  \alpha_{Y} \circ \funcf{F}^{\vee}(f,U)
  &= \Tr{\funcf{F}U}_{\funcf{F}X,\funcf{G}Y}
  ((\alpha_{Y} \otimes \funcf{F}U) \circ
  (m^{\funcf{F}}_{Y,U})^{-1} \circ \funcf{F}f \circ m^{\funcf{F}}_{X,U}) \\
  &= \Tr{\funcf{G}U}_{\funcf{F}X,\funcf{G}Y}((\alpha_{Y} \otimes \alpha_{U}) \circ
  (m^{\funcf{F}}_{Y,U})^{-1} \circ \funcf{F}f \circ m^{\funcf{F}}_{X,U}
  \circ (\funcf{F}X \otimes \alpha^{-1}_{U})) \\
  &= \Tr{\funcf{G}U}_{\funcf{F}X,\funcf{G}Y}((m^{\funcf{G}}_{Y,U})^{-1} \circ
  \funcf{G}f \circ m^{\funcf{G}}_{X,U}
  \circ (\alpha_{X} \otimes \funcf{G}U)) \\
  &= \funcf{G}^{\vee} (f,U) \circ \alpha_{X}.
\end{align*}

\begin{proposition}
  \((-)^{\vee}\) is the inverse of \(\U(-) \circ \embed{A}\).
\end{proposition}
\begin{proof}
  For \(\funcf{F} \colon \catf{A} \to \U\catf{B}\),
  \begin{align*}
    (\U (\funcf{F}^{\vee}) \circ \embed{A})(f \colon X \to Y)
    &= \Tr{\funcf{F}\unitobj}_{\funcf{F}X,\funcf{F}Y} ((m^{\funcf{F}}_{Y,\unitobj})^{-1} \circ
    \funcf{F}(f \otimes \unitobj) \circ m^{\funcf{F}}_{X,\unitobj}) \\
    &= \funcf{F}f.
  \end{align*}
  For \(\funcf{G} \colon \F\catf{A} \to \catf{B}\),
  \begin{align*}
    ((\U \funcf{G}) \circ \embed{A})^{\vee}((f,U) \colon X \to Y)
    &= \Tr{\funcf{G}U}_{\funcf{G}X,\funcf{G}Y}((m^{\funcf{G}}_{Y,U})^{-1}
    \circ \funcf{G}(f \otimes \unitobj,\unitobj) \circ m^{\funcf{G}}_{X,U}) \\
    &= \funcf{G}(\hide{U}(f \otimes \unitobj,\unitobj)) \\
    &= \funcf{G}(f,U).
  \end{align*}
  It is straightforward to check
  \(\U (\alpha^{\vee}) \circ \embed{A} = \alpha\) for any \(\alpha
  \colon \funcf{F} \Rightarrow \funcf{G} \colon \catf{A} \to
  \U\catf{B}\) and \(((\U \beta) \circ \embed{A})^{\vee} = \beta\) for
  any \(\beta \colon \funcf{F} \Rightarrow \funcf{G} \colon \F\catf{A} \to \catf{B}\).
\end{proof}

\paragraph{Failure of a Generalization}

The forgetful \(2\)-functor \(\U' \colon \TSMC \to \SMC\) is no longer
a right \(2\)-adjunction where \(\TSMC\) is the \(2\)-category of traced
symmetric monoidal categories, traced symmetric monoidal functors and
\emph{monoidal natural transformations}, and \(\SMC\) is the
\(2\)-category of symmetric monoidal categories, strong symmetric
monoidal functors and \emph{monoidal natural transformations}.

We suppose that \(\U'\) has a left \(2\)-adjunction \(\F' \colon \SMC
\to \TSMC\) and derive a contradiction. If \(\F'\) is left
\(2\)-adjoint to \(\U'\), then there is a family of isomorphisms
\(\funcf{J}_{\catf{A}} \colon \F\catf{A} \to \F'\catf{A}\) such that
\(\funcf{J}_{\catf{A}} \circ \embed{A}\) is equal to the unit
of \(\F' \dashv \U'\) for any symmetric monoidal category
\(\catf{A}\). In particular,
\begin{equation*}
  \TSMC(\F\Pfn,\F\Pfn) \xrightarrow{\U'}
  \TSMC(\U'\F\Pfn,\U'\F\Pfn) \xrightarrow{(-) \circ \embed{\Pfn}}
  \SMC(\Pfn,\U'\F\Pfn)
\end{equation*}
is an isomorphism where \(\Pfn\) is the symmetric monoidal category of
sets and partial functions and the disjoint sum of sets. Let \(N
\colon \Pfn \to \U'\F(\Pfn)\) be a strong symmetric monoidal functor
given by \(\funcf{N}(X) = \emptyset\). We note that \(\funcf{N}^{\vee} \colon \F\Pfn
\to \F\Pfn\) is also a constant functor given by \(\funcf{N}^{\vee}(X) =
\emptyset\). For a set \(X\), we define \(z_{X} \colon X \to
\emptyset\) to be the empty partial function. Then \((z_{X} +
\emptyset,\emptyset) \colon X \to \emptyset\) is a monoidal natural
transformation from \(\embed{\Pfn}\) to \(\funcf{N}\). Because \(\U(-) \circ
\embed{\Pfn}\) is an isomorphism, \((z_{X} + \emptyset,\emptyset)
\colon X \to \emptyset\) is also a monoidal natural transformation
from \(\Id_{\F\Pfn}\) to \(\funcf{N}^{\vee}\). Hence, the following diagram
commutes:
\begin{equation*}
  \xymatrix@C=15mm{
    \{\ast\} \ar[r]^{\Tr{\NN}_{\{\ast\},\{\ast\}}(f,\NN)} \ar[d]_{z_{\{\ast\}}} &
    \{\ast\} \ar[d]^{z_{\{\ast\}}} \\
    \emptyset \ar@{=}[r] &
    \emptyset
  }
\end{equation*}
where \(f \colon \{\ast\} + \NN \to \{\ast\} + \NN\) is given by
\begin{align*}
  f(\ast) &= 0, \\
  f(n) &= n+1.
\end{align*}
However, we have \(z_{\{\ast\}} \circ \Tr{\NN}_{\{\ast\},\{\ast\}}(f) \neq z_{\{\ast\}}\)
as we observed in (...).
