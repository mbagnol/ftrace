%%% TeX-master: "MAIN" %% Please do not remove this line (hoshino)

\section{\(2\)-adjunction}
\label{2-adjunction}

Let \(\SMCiso\) be the \(2\)-category of symmetric monoidal categories,
strong symmetric monoidal functors and monoidal natural
isomorphisms; and let \(\TSMCiso\) be the \(2\)-category of traced
symmetric monoidal categories, traced symmetric monoidal functors and
monoidal natural isomorphisms. We prove that
\(\U\) is a right \(2\)-adjoint. To see this, we show that
\begin{equation*}
  \TSMCiso(\F\catf{A},\catf{B}) \xrightarrow{\U}
  \TSMCiso(\U\F\catf{A},\U\catf{B}) \xrightarrow{(-) \circ \embed{A}}
  \SMCiso(\catf{A},\U(\catf{\catf{B}})).
\end{equation*}
is an isomorphism.

For a symmetric monoidal functor \(F \colon \catf{A} \to \U(\catf{B})\),
we define a traced symmetric monoidal functor \(F^{\vee} \colon \F\catf{A} \to \catf{B}\)
as follows:
\begin{align*}
  F^{\vee}(X) &= F, \\
  F^{\vee}((f,U) \colon X \to Y)
  &= \Tr{FU}_{FX,FY}((m_{Y,U}^{F})^{-1} \circ Ff \circ m_{X,U}), \\
  n^{F^{\vee}} &= n^{F} \colon I \to F, \\
  m^{F^{\vee}}_{X,Y} &= m^{F}_{X,Y} \colon FX \otimes FY \to F(X \otimes Y).
\end{align*}
For a monoidal natural isomorphism \(\alpha \colon F \to G\), we
define a monoidal natural isomorphism \(\alpha^{\vee} \colon F \to G\)
to be \(\alpha\).

Below, we check that \((-)^{\vee}\) is a functor and \(\alpha^{\vee}\)
is a monoidal natural isomorphism.
We first check that \(F^{\vee}\) is a functor. For \(X \in \catf{A}\),
\begin{align*}
  F^{\vee}(\Id_{X \otimes I},I)
  &= \Tr{FI}_{FX,FX}(\Id_{FX \otimes FI}) \\
  &= \Tr{I}_{FX,FX}(\Id_{FX \otimes I}) \\
  &= \Id_{FX}.
\end{align*}
For \(\F\catf{A}\)-morphisms \((f,U) \colon X \to Y\) and \((g,V)
\colon Y \to Z\), we denote \((m_{Y,U}^{F})^{-1} \circ Ff \circ
m_{X,U}\) by \(f'\) and \((m_{Z,V}^{F})^{-1} \circ Fg \circ m_{Y,V}\)
by \(g'\). Then
\begin{align*}
  F^{\vee}(g,V) \circ F^{\vee}(f,U)
  &= \Tr{FV}_{FY,FZ}(g') \circ \Tr{FU}_{FX,FY}(f') \\
  &= \Tr{FU \otimes FV}_{FX,FZ}(FY \otimes \symm{FV}{FU}) \circ
  (g' \otimes FU) \circ (FY \otimes \symm{FU}{FV}) \circ (f' \otimes FV)) \\
  &= \Tr{FU \otimes FV}_{FX,FZ}(((m_{Z,V}^{F})^{-1} \otimes FU) \circ
  (m_{Z \otimes V,U}^{F})^{-1} \circ
  Fh \circ m_{X \otimes U,V}^{F} \circ (m_{X,U}^{F} \otimes FV)) \\
  &= \Tr{F(U \otimes V)}_{FX,FZ}((m_{Z,U \otimes V}^{F})^{-1} \circ
  Fh \circ m_{X, U \otimes V}^{F}) \\
  &= F^{\vee}((g,V) \circ (f,U))
\end{align*}
where \(h \colon X \otimes U \otimes V \to Z \otimes U \otimes V\) is given by
\begin{equation*}
  (Z \otimes \symm{V}{U}) \circ (g \otimes U) \circ (Y \otimes \symm{U}{V})
  \circ (f \otimes V).
\end{equation*}
We next check naturality of \(m^{F^{\vee}}_{X,Y}\). For
\(\F\catf{A}\)-morphisms \((f,U) \colon X \to Z\) and \((g,V) \colon Y
\to W\), we denote \((m^{F}_{Z,U})^{-1} \circ Ff \circ m^{F}_{X,U}\) by
\(f'\) and \((m_{W,V}^{F})^{-1} \circ Fg \circ m^{F}_{Y,V}\) by \(g'\). Then
\begin{align*}
  &m^{F}_{Z,W} \circ (F^{\vee}(f,U) \otimes F^{\vee}(g,V)) \\
  &= \Tr{FU \otimes FV}_{FX \otimes FY,F(Z \otimes W)}
  ((m^{F}_{Z,W} \otimes FU \otimes FV) \circ
  (FZ \otimes \symm{FU}{FW} \otimes FV) \circ
  (Ff' \otimes Fg') \circ
  (FX \otimes \symm{FU}{FY} \otimes FV)) \\
  &= \Tr{F(U \otimes V)}_{FX \otimes FY,F(Z \otimes W)}
  ((m_{Z \otimes W, U \otimes V}^{F})^{-1} \circ
  Fh \circ
  m_{X \otimes Y, U \otimes V}^{F} \circ
  (m_{X,Y}^{F} \otimes F(U \otimes V))) \\
  &= F^{\vee}((f,U) \otimes (g,V)) \circ m^{F}_{X,Y}
\end{align*}
where \(h \colon X \otimes Y \otimes U \otimes V \to Z \otimes W \otimes U \otimes V\)
is given by
\begin{equation*}
  h = (Z \otimes \symm{U}{W} \otimes V) \circ (f \otimes g) \circ
  (X \otimes \symm{Y}{U} \otimes V).
\end{equation*}
Preservation of trace operators by \(F^{\vee}\) follows from Vanishing
II. Coherence conditions for \(m^{F^{\vee}}\) follows from those for
\(m^{F}\). Lastly, we check that \(\alpha^{\vee}\) is a monoidal
natural isomorphism. Since coherence conditions for \(\alpha^{\vee}\)
follow from those for \(\alpha\), we only check naturality of
\(\alpha^{\vee}\). For a \(\F\catf{A}\)-morphism \((f,U) \colon X \to
Y\),
\begin{align*}
  \alpha_{Y} \circ F^{\vee}(f,U)
  &= \Tr{FU}_{FX,GY}((\alpha_{Y} \otimes FU) \circ (m^{F}_{Y,U})^{-1} \circ Ff \circ m^{F}_{X,U}) \\
  &= \Tr{GU}_{FX,GY}((\alpha_{Y} \otimes \alpha_{U}) \circ (m^{F}_{Y,U})^{-1} \circ Ff \circ m^{F}_{X,U}
  \circ (FX \otimes \alpha^{-1}_{U})) \\
  &= \Tr{GU}_{FX,GY}((m^{G}_{Y,U})^{-1} \circ Gf \circ m^{G}_{X,U}
  \circ (\alpha_{X} \otimes GU)) \\
  &= G^{\vee} (f,U) \circ \alpha_{X}.
\end{align*}

\begin{proposition}
  \((-)^{\vee}\) is the inverse of \(\U(-) \circ \embed{A}\).
\end{proposition}
\begin{proof}
  For \(F \colon \catf{A} \to \U\catf{B}\),
  \begin{align*}
    (\U (F^{\vee}) \circ \embed{A})(f \colon X \to Y)
    &= \Tr{FI}_{FX,FY} ((m^{F}_{Y,I})^{-1} \circ F(f \otimes I) \circ m^{F}_{X,I}) \\
    &= Ff.
  \end{align*}
  For \(G \colon \F\catf{A} \to \catf{B}\),
  \begin{align*}
    ((\U G) \circ \embed{A})^{\vee}((f,U) \colon X \to Y)
    &= \Tr{GU}_{GX,GY}((m^{G}_{Y,U})^{-1} \circ G(f \otimes I,I) \circ m^{G}_{X,U}) \\
    &= G(\hide{U}(f \otimes I,I)) \\
    &= G(f,U).
  \end{align*}
  It is straightforward to check
  \(\U (\alpha^{\vee}) \circ \embed{A} = \alpha\)
  for any \(\alpha \colon F \Rightarrow G \colon \catf{A} \to \U\catf{B}\)
  and \(((\U \beta) \circ \embed{A})^{\vee} = \beta\) 
  for any \(\beta \colon F \Rightarrow G \colon \F\catf{A} \to \catf{B}\).
\end{proof}

\paragraph{Failure of a Generalization}

The forgetful \(2\)-functor \(\U' \colon \TSMC \to \SMC\) is no longer
a right \(2\)-adjunction where \(\TSMC\) is the \(2\)-category of traced
symmetric monoidal categories, traced symmetric monoidal functors and
\emph{monoidal natural transformations}, and \(\SMC\) is the
\(2\)-category of symmetric monoidal categories, strong symmetric
monoidal functors and \emph{monoidal natural transformations}.

We suppose that \(\U'\) has a left \(2\)-adjunction \(\F'\) and derive
a contradiction. If \(\F'\) is left \(2\)-adjoint to \(\U'\), then
\(\F'\catf{A}\) is isomorphic to \(\F\catf{A}\) for any traced
symmetric monoidal categories \(\catf{A}\). Therefore,
\begin{equation*}
  \TSMC(\F\Pfn,\F\Pfn) \xrightarrow{\U}
  \TSMC(\U'\F\Pfn,\U'\F\Pfn) \xrightarrow{(-) \circ \embed{\Pfn}}
  \SMC(\Pfn,\U'\F\Pfn)
\end{equation*}
is an isomorphism where \(\Pfn\) is the category of sets and partial
functions regarded as a symmetric monoidal category by taking the
disjoint sum of sets as its monoidal product. Let \(N \colon \Pfn \to
\F(\Pfn)\) be a strong symmetric monoidal functor given by \(N(X) =
\emptyset\). For a set \(X\), we define \(z_{X} \colon X \to
\emptyset\) to be the empty partial function. Because \(z_{X}\) is a
monoidal natural transformation from \(\embed{\Pfn}\) to \(N\) and
\(\U(-) \circ \embed{\Pfn}\) is an isomorphism, \(z_{X}\) is also a
monoidal natural transformation from \(\Id_{\F\Pfn}\) to \(N^{\vee}\).
By the definition of \((-)^{\vee}\), we see that \(N^{\vee}(f \colon X
\to Y)\) is equal to the empty partial function from \(X\) to \(Y\)
for any partial function \(f \colon X \to Y\). By naturality of
\(z_{X}\), the following diagram commutes:
\begin{equation*}
  \xymatrix{
    1 \ar[r]^{\Tr{\NN}_{1,1}(f)} \ar[d]_{z_{1}} &
    1 \ar[d]^{z_{1}} \\
    \emptyset \ar@{=}[r] &
    \emptyset
  }
\end{equation*}
where \(f \colon \{\ast\} + \NN \to \{\ast\} + \NN\) is given by
\begin{align*}
  f(\ast) &= 0, \\
  f(n) &= n+1.
\end{align*}
However, we have \(z_{1} \circ \Tr{\NN}_{1,1}(f) \neq z_{1}\) as we observed in (...).
