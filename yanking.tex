\begin{itemize}
	\item adding the missing equation via a further quotient
	\item free trace
\end{itemize}


%We now want to enforce the yanking equation on $\hide{}[\cdot]$ in $\dial C$. To do so, we follow 
%the same approach as
%before and further quotient $\dial C$.

%\begin{definition}[trace category]
%	Define the following relation on $\state C$ morphisms:
%	$$\big(\,(f\monp V)\sq(B\monp\symm VV)\sq(g\monp V)\,,\, U\monp V\,\big)
%	\yr V (f\sq g\,,\,U) $$
%	and set $f\yl{}g=g\yr{}f$. These are then lifted to $\dial C$ morphisms by saying that two
%	equivalence class are related whenever they contain two related elements.
%	
%	We obtain two equivalence relations:
%	\begin{itemize}
%		\item In $\dial C$, the transitive closure of $\yr{}\cup\yl{}$, which we write $\ye$.
%		\item In $\state C$, the transitive closure of $\eq\cup\yr{}\cup\yl{}$, which we write $\quo$.
%	\end{itemize}
%	The \emph{trace category} over $\catf C$ is defined as $\tra C=\quotient{\dial C}{\ye}
%	=\quotient{\state C}{\quo}$,
%	that is the category $\dial C$ where morphisms are further equated by $\ye$ or equivalently
%	the pseudo-category $\state C$ with morphisms equated by $\quo$.
%\end{definition}

%\begin{proposition}
%	The above defined $\tra C$ is indeed a category and
%	the symmetric monoidal structure of $\dial C$ lifts to $\tra C$.
%\end{proposition}

%\begin{proof}
%	Amounts to showing that $\yr{}$ is a \enquote{monoidal congruence}, graphical computations
%	make this fairly straightforward.
%\end{proof}

%\begin{proposition}
%	There is a monoidal functor $\embt C$ from $C$ to $\tra C$, defined as
%	$\embt C(f)=(f,\unitobj)$.
%\end{proposition}

%\begin{remark}
%	As we observed, this is not necessarily an embedding.
%\end{remark}

%\begin{theorem}
%	The hiding operation $\hide{}[\cdot]$ lifts to $\tra C$ and makes it a traced category.
%	
%	Moreover, $\tra C$ satisfies the following universal property: for any traced
%	category $\catf D$ and monoidal functor $\morph F{\catf C}{\catf D}$ we have that $F$
%	factors uniquely as 
%	\begin{center}
%	$\funcf F=\funcf G\circ \embt C$ \qquad\qquad
%	\begin{tikzpicture}[baseline=-5ex]
%		\node (c) at (0,0) {$\catf C$};
%		\node (tc) at (2,0) {$\tra C$};
%		\node (d) at (2,-1.5) {$\catf D$};
%		\draw[->] (c) to node [above] {$\embt C$} (tc);
%		\draw[->] (c) to node [below left] {$\funcf F$} (d);
%		\draw[->,densely dotted] (tc) to node [right] {$\funcf G$} (d);
%	\end{tikzpicture}
%	\end{center}
%	with $G$ a traced functor.
%\end{theorem}

%\begin{proof}
%	Same as in the previous section: define $G(f,U)=\Tr{\funcf FU}[\funcf F f]$.
%\end{proof}


%\begin{remark}
%	As before, this implies that $\tra\cdot$ is a functor. We could probably define in the same 
%	spirit an operation that takes a pseudo-traced category as an input (not necessarily the
%	result of applying $\dial\cdot$) and freely makes it a traced category.
%\end{remark}

