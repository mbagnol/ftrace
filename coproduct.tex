%%% TeX-master: "MAIN" %% Please do not remove this line (hoshino)

\section{Building the Free Trace of Monoidal Categories with Coproducts}
\subsection{Definitions}

\begin{definition}
  A category $\mathcal{C}$ is \emph{pointed} when $\mathcal{C}$
  has a zero object.
\end{definition}

\begin{definition}
  A \emph{distributive} symmetric monoidal category is a symmetric monoidal
  category with finite coproducts such that canonical morphisms
  \begin{align*}
    0 &\longrightarrow X \otimes 0 \\
    (X \otimes Y) \oplus (X \otimes Z) &\xrightarrow{\mathrm{dst}_{X,Y,Z}} X \otimes (Y \oplus Z)
  \end{align*}
  are isomorphisms.
\end{definition}

In pointed distributive symmetric monoidal categories, we have ``projections'':
\begin{align*}
  \pi_{X,Y} &= X \oplus Y \xrightarrow{X \oplus !_{Y}} X \oplus 0 \xrightarrow{\cong} X, \\
  \pi'_{X,Y} &= X \oplus Y \xrightarrow{!_{X} \oplus Y} 0 \oplus Y \xrightarrow{\cong} Y.
\end{align*}
They are natural transforamtions, and it is easy to check
\begin{align*}
  \iota_{X,Y} \fatsemi \pi_{X,Y} &= \Id_{X}, &
  \iota_{X,Y} \fatsemi \pi'_{X,Y} &= X \to 0 \to Y, &
  \pi_{X,Y} \fatsemi \iota_{X,Y} &= X \oplus (Y \to 0 \to Y), \\
  \iota'_{X,Y} \fatsemi \pi'_{X,Y} &= \Id_{Y}, &
  \iota'_{X,Y} \fatsemi \pi_{X,Y} &= Y \to 0 \to X, &
  \pi'_{X,Y} \fatsemi \iota'_{X,Y} &= (X \to 0 \to X) \oplus Y
\end{align*}
where $\iota_{X,Y} \colon X \to X \oplus Y$ and
$\iota'_{X,Y} \colon Y \to X \oplus Y$ are injections. 

\begin{definition}
  A category $\mathcal{C}$ has \emph{(finite) biproducts} when
  $\mathcal{C}$ has finite coproducts and every hom-set
  $\mathcal{C}$ has the structure of a commutative monoid and
  composition is bilinear.
\end{definition}

\begin{proposition}
  Let $F \colon \mathcal{C} \to \mathcal{D}$ be a functor between
  categories with biproducts. If $F$ preserves finite coproducts,
  then $F$ preserves commutative monoid structures on hom-sets.
\end{proposition}

\section{Goal}

We show that
the construction $\mathfrak{T}$ with a symmetric monoidal functor
$H \colon \mathcal{C} \to \mathfrak{T}(\mathcal{C})$ given by
\begin{equation*}
  H(f \colon X \to Y) = [I,f] \colon X \to Y
\end{equation*}
is left 2-adjoint to the forgetful functor
$\mathfrak{U} \colon \mathfrak{DTsmc} \to \mathfrak{Dsmc}$
where $\mathfrak{DTsmc}$ is the 2-category consisting of
\begin{itemize}
\item traced distributive symmetric monoidal categories with
  finite coproducts,
\item finite coproduct preserving traced symmetric monoidal functors,
\item monoidal natural isomorphisms,
\end{itemize}
and $\mathfrak{Dsmc}$ is the 2-category consisting of
\begin{itemize}
\item pointed distributive symmetric monoidal categories,
\item finite coproduct preserving symmetric monoidal functors,
\item monoidal natural isomorphisms.
\end{itemize}

\section{Existence of Finite Coproducts}

\subsection{Commutative Monoid Enrichement}

Let $\mathbf{CMon}$ be the category of commutative monoids and
homomorphisms. A \emph{$\mathbf{CMon}$-enriched symmetric monoidal
  category} is a symmetric monoidal category such that every
hom-set has the structure of a commutative monoid and the composition
operation and the monoidal product operation are bilinear.
\begin{theorem}
  If $(\mathcal{C},I,\otimes)$ is a pointed distributive symmetric
  monoidal category, then $\mathfrak{T}(\mathcal{C})$ has finite
  coproducts.
\end{theorem}

In the remaining part of this section, we give a proof of this theorem.
We first introduce addition of $\mathfrak{T}(\mathcal{C})$-morphisms.
For $\mathfrak{T}(\mathcal{C})$-morphisms $[U,f] \colon X \to Y$
and $[V,g] \colon X \to Y$, we define a $\mathfrak{T}(\mathcal{C})$-morphism
$[U,f] + [V,g] \colon X \to Y$ by
\begin{equation*}
  [U,f] + [V,g] = [U \oplus V, \mathrm{dst}_{X,U,V}^{-1}
  \fatsemi (f \oplus g) \fatsemi \mathrm{dst}_{Y,U,V}].
\end{equation*}
Below, we write
$f \oplus g \colon X \otimes (U \oplus V) \to Y \otimes (U \oplus
V)$ to indicate
$\mathrm{dst}_{X,U,V}^{-1} \fatsemi (f \oplus g) \fatsemi
\mathrm{dst}_{Y,U,V}$.


\paragraph{Well-definedness of Addition}

We check that if $(U,f) \sim (U',f')$, then
\begin{equation*}\label{eq:welldef}
  X \otimes \hide{(U \oplus V)} \xrightarrow{f \oplus g}
  Y \otimes \hide{(U \oplus V)}
  \sim X \otimes \hide{(U' \oplus V)}
  \xrightarrow{f' \oplus g} Y \otimes \hide{(U' \oplus V)}.
\end{equation*}
To see this, it is enough to check
the following two cases.
\begin{itemize}
\item $f = (X \otimes h) \fatsemi p$ and
  $f' = p \fatsemi (Y \otimes h)$ for some $h \colon U \to U'$ and
  $p \colon X \otimes U' \to Y \otimes U$.
% \item $U = W \otimes 0$ and
%   $f = (p \otimes 0) \fatsemi (Z \otimes
%   \mathrm{sym}_{0,0}) \fatsemi (q \otimes 0)$ and
%   $f' = p \fatsemi q$ for some
%   $p \colon X \otimes W \to Z \otimes 0$ and
%   $q \colon Z \otimes 0 \to Y \otimes W$.
\item $U = U_{0} \otimes U_{1}$ and
  $f = (p \otimes U_{1}) \fatsemi (Z \otimes
  \mathrm{sym}_{U_{1},U_{1}}) \fatsemi (q \otimes U_{1})$ and
  $f' = p \fatsemi q$ for some % $U_{1} \not\cong 0$ and
  $p \colon X \otimes U_{0} \to Z \otimes U_{1}$ and
  $q \colon Z \otimes U_{1} \to Y \otimes U_{0}$.
\end{itemize}
We can check the first case as follows.
\begin{align*}
  &X \otimes \hide{(U \oplus V)} \xrightarrow{f \oplus g}
  Y \otimes \hide{(U \oplus V)} \\
  &= X \otimes \hide{(U \oplus V)} \xrightarrow{X \otimes (h \oplus V)}
    X \otimes (U' \oplus V) \xrightarrow{p \oplus g}
    Y \otimes \hide{(U \oplus V)} \\
  &\sim X \otimes \hide{(U' \oplus V)} \xrightarrow{p \oplus g}
    Y \otimes (U \oplus V) \xrightarrow{Y \otimes (h \oplus V)}
    Y \otimes \hide{(U' \oplus V)} \\
  &= X \otimes \hide{(U' \oplus V)} \xrightarrow{f' \oplus g}
    Y \otimes \hide{(U' \oplus V)}.
\end{align*}
Next, we check the second case. Let
$h \colon X \otimes (U_{0} \oplus V) \otimes (U_{1} \oplus I) \to Y
\otimes (U_{0} \oplus V) \otimes (U_{1} \oplus I)$ be
\begin{equation*}
  (r \otimes (U_{1} \oplus I)) \fatsemi
  ((Z \oplus (Y \otimes V)) \otimes \mathrm{sym}_{U_{1} \oplus I,U_{1} \oplus I})
  \fatsemi (s \otimes (U_{1} \oplus I))
\end{equation*}
where $r \colon X \otimes (U_{1} \oplus V) \to (Z \oplus (Y \otimes V))
\otimes (U_{0} \oplus I)$ and
$s \colon (Z \oplus (Y \otimes V)) \otimes (U_{0} \oplus I)
\to Y \otimes (U_{1} \oplus V) \otimes (U_{0} \oplus I)$ are
given by
\begin{align*}
  r
  &= X \otimes (U_{0} \oplus V) \xrightarrow{\mathrm{dst}^{-1}}
    (X \otimes U_{0}) \oplus (X \otimes V) \xrightarrow{p \oplus g}
    (Z \otimes U_{1}) \oplus (Y \otimes V) \xrightarrow{\mathrm{inj}} \\
  &\hspace{6em}
    (Z \otimes U_{1}) \oplus Z \oplus (Y \otimes V \otimes U_{1})
    \oplus (Y \otimes V)\xrightarrow{\mathrm{dst}}
    (Z \oplus (Y \otimes V)) \otimes (U_{1} \oplus I), \\
  s
  &= (Z \oplus (Y \otimes V)) \otimes (U_{1} \oplus I)
    \xrightarrow{\mathrm{dst}^{-1}}
    (Z \otimes U_{1}) \oplus Z \oplus (Y \otimes V \otimes U_{1}) \oplus (Y \otimes V)
    \xrightarrow{\mathrm{proj}} \\
  &\hspace{6em} (Z \otimes U_{1}) \oplus (Y \otimes V)
    \xrightarrow{q \oplus (Y \otimes V)}
    (Y \otimes U_{0}) \oplus (Y \otimes V)
    \xrightarrow{\mathrm{dst}}
    Y \otimes (U_{0} \oplus V).
\end{align*}
Because $r \fatsemi s = (p \fatsemi q) \oplus g$, we have
\begin{align*}
  &X \otimes \hide{(U_{0} \oplus V)} \xrightarrow{f' + g}
    Y \otimes \hide{(U_{0} \oplus V)} \\
  &= X \otimes \hide{(U_{0} \oplus V)} \xrightarrow{r \fatsemi s}
    Y \otimes \hide{U_{0} \oplus V} \\
  &\sim
    X \otimes \hide{(U_{0} \oplus V) \otimes (U_{1} \oplus V)}
    \xrightarrow{h} Y \otimes \hide{(U_{0} \oplus V) \otimes (U_{1} \oplus I)}.
\end{align*}
We define $\mathcal{C}$-morphisms $d$ and $e$ by
\begin{align*}
  d &= (U_{0} \otimes U_{1}) \oplus V \xrightarrow{\mathrm{inj}}
      (U_{0} \otimes U_{1}) \oplus U_{0} \oplus (V \otimes U_{1}) \oplus V
      \xrightarrow{\mathrm{dst}}
      (U_{1} \oplus V) \otimes (U_{0} \oplus I), \\
  e &= (U_{1} \oplus V) \otimes (U_{0} \oplus I)
      \xrightarrow{\mathrm{dst}^{-1}}
      (U_{0} \otimes U_{1}) \oplus U_{0} \oplus (V \otimes U_{1}) \oplus V
      \xrightarrow{\mathrm{proj}} (U_{0} \otimes U_{1}) \oplus V.      
\end{align*}
Because $d \fatsemi e = \mathrm{id}_{(U_{0} \otimes U_{1}) \oplus V}$ and
\begin{math}
  h = (X \otimes (e \fatsemi d)) \fatsemi h,
\end{math}
we obtain
\begin{align*}
  [U',f'] + [V,g] 
  &= [(U_{0} \oplus V) \otimes (U_{1} \oplus I),h] \\
  &= [(U_{0} \otimes U_{1}) \oplus V,(X \otimes d) \fatsemi h \fatsemi (X \otimes e)] \\
  &= [U,f] + [V,g].
\end{align*}
We can similarly check that $[U,f] + [V,g]$ is independent of
representative of $[V,g]$.

\paragraph{Unit Laws}

For $[U,f] \colon X \to Y$, because the left injection
$\mathrm{inj} \colon U \to U \oplus 0$ is an isomorphism, we get
\begin{align*}
  f &= X \otimes \hide{U} \xrightarrow{X \otimes \mathrm{inj}}
  X \otimes (U \oplus 0) \xrightarrow{f \oplus o_{X,Y}}
  Y \otimes (U \oplus 0) \xrightarrow{Y \otimes \mathrm{inj}^{-1}}
  Y \otimes \hide{U} \\
  &\sim X \otimes \hide{(U \oplus 0)}
  \xrightarrow{f \oplus o_{X,Y}}
  Y \otimes \hide{(U \oplus 0)}.
\end{align*}
Hence, $[U,f] + [0,o_{X,Y}] = [U,f]$. Similarly, we can show that
$[0,o_{X,Y}] + [U,f] = [U,f]$.

\paragraph{Symmetry}

For $[U,f],[V,g] \colon X \to Y$,
\begin{align*}
  & X \otimes \hide{(U \oplus V)} \xrightarrow{f \oplus g}
    Y \otimes \hide{(U \oplus V)} \\
  &= X \otimes \hide{(U \oplus V)} \xrightarrow{X \otimes \mathrm{sym}}
    X \otimes (V \oplus U) \xrightarrow{g \oplus f}
    Y \otimes (V \oplus U) \xrightarrow{Y \otimes \mathrm{sym}}
    Y \otimes \hide{(U \oplus V)} \\
  &\sim X \otimes \hide{(V \oplus U)} \xrightarrow{g \oplus f}
    Y \otimes \hide{(V \oplus U)}.
\end{align*}
Hence, $[U,f] + [V,g] = [V,g] + [U,f]$.

\paragraph{Associativity}

For $[U,f],[V,g],[W,h] \colon X \to Y$, we have
\begin{align*}
  & X \otimes \hide{(U \oplus (V \oplus W))}
    \xrightarrow{f \oplus (g \oplus h)}
    Y \otimes \hide{(U \oplus (V \oplus W))} \\
  &= X \otimes \hide{(U \oplus (V \oplus W))} \xrightarrow{X \otimes \mathrm{assoc}}
    X \otimes ((U \oplus V) \oplus W) \xrightarrow{(f \oplus g) \oplus h} \\
  &\hspace{15em}
    Y \otimes ((U \oplus V) \oplus W) \xrightarrow{Y \otimes \mathrm{assoc}^{-1}}
    Y \otimes \hide{(U \oplus (V \oplus W))} \\
  &\sim
    X \otimes \hide{((U \oplus V) \oplus W)} \xrightarrow{(f \oplus g) \oplus h}
    Y \otimes \hide{((U \oplus V) \oplus W)}.
\end{align*}
Hence, $[U,f] + ([V,g] + [W,h]) = ([U,f] + [V,g]) + [W,h]$.

\paragraph{Bilinearity}

For $[U,f],[V,g] \colon X \to Y$ and $[W,h] \colon Y \to Z$,
\begin{align*}
  & X \otimes \hide{(U \oplus V) \otimes W} \xrightarrow{(f \oplus g) \bullet h}
    Z \otimes \hide{(U \oplus V) \otimes W} \\
  &= X \otimes \hide{(U \oplus V) \otimes W} \xrightarrow{X \otimes \mathrm{dst}^{-1}}
    X \otimes ((U \otimes W) \oplus (V \otimes W)) \xrightarrow{(f \bullet h) \oplus (g \bullet h)}
     \\
  &\hspace{15em}
    Z \otimes ((U \otimes W) \oplus (V \otimes W))
    \xrightarrow{Z \otimes \mathrm{dst}}
    Z \otimes \hide{(U \oplus V) \otimes W} \\
  &\sim
    X \otimes \hide{((U \otimes W) \oplus (V \otimes W))}
    \xrightarrow{(f \bullet h) \oplus (g \bullet h)}
    Z \otimes \hide{((U \otimes W) \oplus (V \otimes W))}
\end{align*}
where
\begin{equation*}
  (-) \bullet h =
  ((-) \otimes V) \fatsemi (Y \otimes \mathrm{sym}_{U,V})
  \fatsemi (h \otimes U) \fatsemi (Z \otimes \mathrm{sym}_{V,U}).
\end{equation*}
Hence, $[W,h] \circ ([U,f] + [V,g]) = [W,h] \circ [U,f] + [W,h] \circ
[V,g]$. We can similarly check $[W,h] \circ [0,o_{X,Y}] = [0,o_{X,Z}]$
using $0 \otimes W \cong 0$. Linearity of $[W,h] \otimes (-)$
and $[W,h] \otimes (-)$ are checked in the same way.

\subsection{Existence of Coproducts}

\begin{lemma}
  $[I,\pi_{X,Y} \fatsemi \iota_{X,Y}] +
  [I,\pi'_{X,Y} \fatsemi \iota'_{X,Y}] = \Id_{X \oplus Y}$.
\end{lemma}
\begin{proof}
  We define $u \colon X \oplus Y \to (X \oplus Y) \otimes (I \oplus I)$
  and $v \colon (X \oplus Y) \otimes (I \oplus I) \to X \oplus Y$ by
  \begin{align*}
    u &= X \oplus Y \xrightarrow{\iota_{X,Y} \oplus \iota'_{X,Y}}
        (X \oplus Y) \oplus (X \oplus Y) \xrightarrow{\mathrm{dst}}
        (X \oplus Y) \otimes (I \oplus I), \\
    v &= (X \oplus Y) \otimes (I \oplus I) \xrightarrow{\mathrm{dst}^{-1}}
        (X \oplus Y) \oplus (X \oplus Y) \xrightarrow{\pi_{X,Y} \oplus \pi'_{X,Y}}
        X \oplus Y.
  \end{align*}
  Then
  \begin{align*}
    & (X \oplus Y) \otimes \hide{(I \oplus I)}
      \xrightarrow{(\pi_{X,Y} \fatsemi \iota_{X,Y}) \oplus (\pi'_{X,Y} \fatsemi \iota'_{X,Y})}
      (X \oplus Y) \otimes \hide{(I \oplus I)} \\
    &= (X \oplus Y) \otimes \hide{(I \oplus I)}
      \xrightarrow{u \otimes (I \oplus I)}
      (X \oplus Y) \otimes (I \oplus I) \otimes (I \oplus I)
      \xrightarrow{(X \oplus Y) \otimes \mathrm{sym}} \\
    &\hspace{10em}
      (X \oplus Y) \otimes (I \oplus I) \otimes (I \oplus I)
      \xrightarrow{v \otimes (I \oplus I)}
      (X \oplus Y) \otimes \hide{(I \oplus I)} \\
    &\sim
      (X \oplus Y) \otimes \hide{I}
      \xrightarrow{u \fatsemi v}
      (X \oplus Y) \otimes \hide{I} \\
    &=
      (X \oplus Y) \otimes \hide{I}
      \xrightarrow{\Id_{X \oplus Y}}
      (X \oplus Y) \otimes \hide{I}.
  \end{align*}
\end{proof}

\begin{theorem}\label{thm:biproduct}
  $\mathfrak{T}(\mathcal{C})$ has finite coproducts.
\end{theorem}
\begin{proof}
  We have already checked that $\mathfrak{T}(\mathcal{C})$
  has a zero object and $\mathfrak{T}(\mathcal{C})$ is
  $\mathbf{CMon}$-enriched. Hence, it remains to check
  existence of binary coproducts in $\mathfrak{T}(\mathcal{C})$.
  For $[U,f] \colon X \to Y$ and $[V,g] \colon Z \to Y$,
  if there is $[W,h] \colon X \oplus Z \to Y$ such that
  $[I,\iota_{X,Z}] \fatsemi [W,h] = [U,f]$ and
  $[I,\iota'_{X,Z}] \fatsemi [W,h] = [V,g]$, then we have
  \begin{align*}
    [W,h]
    &= ([I,\pi_{X,Z}] \fatsemi [I,\iota_{X,Z}] + [I,\pi'_{X,Z}] \fatsemi [I,\iota'_{X,Z}])
      \fatsemi [W,h] \\
    &= [I,\pi_{X,Z}] \fatsemi [U,f] + [I,\pi'_{X,Z}] \fatsemi [V,g].
  \end{align*}
  Because $[I,0_{X,Y}] = [0,o_{X,Y}]$, we have
  \begin{align*}
    [I,\iota_{X,Z}] \fatsemi ([I,\pi_{X,Z}] \fatsemi [U,f] + [I,\pi'_{X,Z}] \fatsemi [V,g])
    &= [U,f] + [I,0_{X,Z}] = [U,f], \\
    [I,\iota'_{X,Z}] \fatsemi ([I,\pi_{X,Z}] \fatsemi [U,f] + [I,\pi'_{X,Z}] \fatsemi [V,g])
    &= [I,0_{X,Z}] + [V,g] = [V,g].
  \end{align*}
  Hence, $(X \oplus Z,[I,\iota_{X,Z}],[I,\iota'_{X,Z}])$ is a
  coproduct of $X$ and $Z$.
\end{proof}

Let $h \colon X \oplus Z \to Y$ be the cotupling of
$\mathcal{C}$-morphisms $f \colon X \to Y$ and
$g \colon Z \to Y$. Then it is easy to see that
$[I,h] \colon X \oplus Z \to Y$ is the cotupling of
$\mathfrak{T}(\mathcal{C})$-morphisms $[I,f] \colon X \to Y$ and
$[I,g] \colon Z \to Y$. Hence, the canonical functor
$H \colon \mathcal{C} \to \mathfrak{T}(\mathcal{C})$ preserves
coproducts on the nose.

\begin{corollary}[Hasegawa]
  The canonical functor $H \colon \mathcal{C} \to \mathfrak{T}(\mathcal{C})$
  preserves finite coproducts.
\end{corollary}

We define $\mathfrak{T}(\mathcal{C})_{\mathrm{fin}}$ to be a symmetric monoidal
full subcategory of $\mathfrak{T}(\mathcal{C})$ generated by
objects of the form $I \oplus \cdots \oplus I$.
\begin{corollary}\label{cor:compact}
  $\mathfrak{T}(\mathcal{C})_{\mathrm{fin}}$ is a compact closed category.
\end{corollary}
\begin{proof}
  We write $2$ for $I \oplus I$.
  We define $\eta \colon I \to 2 \otimes 2$ by
  \begin{equation*}
    \eta = [I,\iota_{I,I}] \otimes [I,\iota_{I,I}] +
    [I,\iota'_{I,I}] \otimes [I,\iota'_{I,I}],
  \end{equation*}
  and we define $\epsilon \colon 2 \otimes 2 \to I$ by
  \begin{equation*}
    \epsilon = [I,\pi_{I,I}] \otimes [I,\pi_{I,I}] +
    [I,\pi'_{I,I}] \otimes [I,\pi'_{I,I}].
  \end{equation*}
  Then we have
  \begin{align*}
    (2 \otimes \eta) \fatsemi (\epsilon \otimes 2)
    &= (2 \otimes [I,\iota_{I,I}] \otimes [I,\iota_{I,I}] +
      2 \otimes [I,\iota'_{I,I}] \otimes [I,\iota'_{I,I}]) \fatsemi {} \\
    &\hspace{10em}
      ([I,\pi_{I,I}] \otimes [I,\pi_{I,I}] \otimes 2 + [I,\pi'_{I,I}] \otimes
      [I,\pi'_{I,I}] \otimes 2) \\
    &= [I,\pi_{I,I}] \fatsemi [I,\iota_{I,I}] + [I,\pi'_{I,I}]
      \fatsemi [I,\iota'_{I,I}] = \Id_{I \oplus I}.
  \end{align*}
  It is straightforward to extend this argument to general situation.
\end{proof}

\begin{remark}
  Corollary~\ref{cor:compact} gives another
  description of the trace operator on $\mathfrak{T}(\mathcal{C})_{\mathrm{fin}}$:
  \begin{equation*}
    \Tr{I^{\oplus l}}_{I^{\oplus n},I^{\oplus m}}([U,f])
    = \sum_{i = 1}^{l}
    (I^{\oplus n} \otimes [I,\iota_{i}]) \fatsemi [U,f] \fatsemi
    (I^{\oplus m} \otimes [I,\pi_{i}])
  \end{equation*}
  where $\iota_{i} \colon I \to I^{\oplus l}$ is the $i$th
  injection and $\pi_{i} \colon I^{\oplus l} \to I$ is the $i$th
  projection.
\end{remark}

\begin{remark}
  For any cardinal $\alpha$,
  if $\mathcal{C}$ has $\alpha$-coproducts, then under the same
  assumption on $\mathcal{C}$, we can similarly show that
  $\mathfrak{T}(\mathcal{C})_{\alpha}$ has $\alpha$-biproducts.
  (yet to be checked)
\end{remark}

\section{2-categorical Characterisation of $\mathfrak{T}$}

To be done...


\section{Concrete descriptions}

\subsection{Partial Functions}

Let $(\fPfn,1,\times)$ be the symmetric monoidal category of finite
sets and partial functions. This category has finite coproducts given
by disjoint sum and is distributive. Since every object
$X \in \fPfn$ is isomorphic to a finite coproduct of the monoidal
unit $1 \in \fPfn$, we obtain
\begin{equation*}
  \mathfrak{T}(\fPfn)(X,Y) \cong
  \mathfrak{T}(\fPfn)(1,1)^{X \times Y}
  \cong \left\{\overbrace{[1,\Id_{1}] + \cdots + [1,\Id_{1}]}^{n}
    \mid n \in \mathbb{N}\right\}^{X \times Y}.
\end{equation*}
By universality of $\mathfrak{T}(\fPfn)$, we obtain
the following commutative diagram:
\begin{equation*}
  \xymatrix{
    \mathfrak{T}(\fPfn) \ar[r]^{F} &
    \mathbf{Mat}(\mathbb{N}) \ar[r]^{G} &
    \mathfrak{T}(\fPfn) \\
    & \fPfn \ar[ru]_{H} \ar[u]_{J} \ar[lu]^{H} &
  }
\end{equation*}
where $\mathbf{Mat}(\mathbb{N})$ is the category of finite numbers
and matrices over the semiring $\mathbb{N}$, and
$G \colon \mathbf{Mat}(\mathbb{N}) \to \mathfrak{T}(\fPfn)$
is a symmetric monoidal functor given by
\begin{equation*}
  G((a_{i,j}) \colon n \to m)
  = ([1,a_{i,j}]) \colon n \to m.
\end{equation*}
Because $GF = \Id$, $F$ is faithful, and
because $J$ and $H$ are bijective on objects,
$F$ is bijective on objects.
It is easy to see that any $\mathbf{Mat}(\mathbb{N})$-morphism
is obtained by finite sum of images of $\fPfn$-morphisms.
Since $F$ is traced, $F$ is epi. Hence, $F$ is an isomorphism.


\subsection{Compact Closed Categories}

Because any symmetric monoidal functor from a compact closed category
to a traced symmetric monoidal category is traced, any compact closed
category $\mathcal{C}$ is isomorphic to
$\mathfrak{T}(\mathcal{C})$. For a few concrete compact closed
categories, we can check this using Theorem~\ref{thm:biproduct}. Let
$(\fRel,1,\times)$ be the symmetric monoidal category of finite sets
and relations. This category is traced and has finite biproducts.
Since $H \colon \fRel \to \mathfrak{T}(\fRel)$ preserves finite
biproducts, we obtain the following commutative diagram:
\begin{equation*}
  \xymatrix{
    \mathfrak{T}(\fRel) \ar[r]^{F} &
    \fRel \ar[r]^{G} &
    \mathfrak{T}(\fRel) \\
    & \fRel \ar[ru] \ar[u]_{\Id} \ar[lu] &
  }
\end{equation*}
Since $HF = \Id$, we see that $F$ is full and faithful and
bijective on objects. Hence, $F$ is an isomorphism. In the same way,
we can show that the category $\mathbf{fdVect}$ of finite
demensional vector spaces and linear functions is isomorphic to
$\mathfrak{T}(\mathbf{fdVect})$.

\subsection{Substochastic Matrix}

An $n$-by-$m$ matrix $F=(F_{i,j})$ over $\mathbb{R}_{\geq 0}$ is
\emph{substochastic} when $\sum_{i=1}^{n}F_{i,j} \leq 1$. We define
a category $\Stc$ by: objects are non-negative integers, and
morphisms from $n$ to $m$ are substochastic matrices.
This category inherits symmetric monoidal structure
and coproducts from the category of non-negative integers
and matrices over $\mathbb{R}$. Since $0$ is a zero object
of $\Stc$, by applying Theorem~\ref{thm:biproduct}, we get
\begin{equation*}
  \mathfrak{T}(\Stc)(X,Y) \cong
  \mathfrak{T}(\Stc)(1,1)^{X \times Y}
  \cong \left\{\overbrace{[1,x_{1}] + \cdots + [1,x_{n}]}^{n}
    \mid 0 \leq x_{i} \leq 1 \right\}^{X \times Y}.
\end{equation*}
For $x,y \in [0,1]$ such that $x+y \leq 1$, we have
\begin{align*}
  [1,x + y]
  &= \left[1,
    \begin{pmatrix}
      1 & 1 \\
    \end{pmatrix}\right]
  \fatsemi
  \left[1,
  \begin{pmatrix}
    x \\
    y \\
  \end{pmatrix}
  \right] \\
  &= \left[1,
    \begin{pmatrix}
      1 & 1 \\
    \end{pmatrix}\right]
  \fatsemi
  \left(
  \left[1,
  \begin{pmatrix}
    x \\
    0 \\
  \end{pmatrix}
  \right] +
  \left[1,
  \begin{pmatrix}
    0 \\
    y \\
  \end{pmatrix}
  \right]\right)\\
  &= [1,x] + [1,y].
\end{align*}
% Hence, for any $x_{1},\ldots,x_{n} \in [0,1]$,
% there is a non-negative integer $m$ and $y \in [0,1]$
% such that
% \begin{equation*}
%   [1,x_{1}] + \cdots + [1,x_{n}]
%   = \overbrace{[1,1] + \cdots + [1,1]}^{m} + [1,y].
% \end{equation*}
Hence, for any $x_{1},\ldots,x_{n} \in [0,1]$
and for any $y_{1},\ldots,y_{m} \in [0,1]$,
if $\sum_{i = 1}^{n}x_{i} = \sum_{i = 1}^{m}y_{i}$,
then
\begin{equation*}
  \sum_{i = 1}^{n}[1,x_{i}] = \sum_{i = 1}^{m}[1,y_{i}].
\end{equation*}
As a consequence, we obtain the following
commutative diagram
\begin{equation*}
  \xymatrix{
    \mathfrak{T}(\Stc) \ar[r]^{F} &
    \mathbf{Mat}(\mathbb{R}_{\geq 0}) \ar[r]^{G} & \mathfrak{T}(\Stc) \\
    & \Stc \ar[lu]^{H} \ar[u]_{J} \ar[ru]_{H} &
  }
\end{equation*}
where $G \colon \mathbf{Mat}(\mathbb{R}_{\geq 0}) \to \mathfrak{T}(\Stc)$
is a symmetric monoidal functor given by
\begin{equation*}
  G((a_{i,j})_{i \in n, j \in m} \colon n \to m) =
  ([1,a_{i,j}])_{i \in n, j \in m}.
\end{equation*}
Because $H$ and $J$ are bijective on objects,
$F$ is bijective on objects. Faithfullness of $F$
follows from $GF=\Id$. Furthremore,
since any $\mathbf{Mat}(\mathbb{R}_{\geq 0})$-morphism
is obtained by finite sum of images of $\Stc$-morphisms,
$F$ is full. Hence, $F$ is an isomorphism.



% We show the other direction:
% $\sum_{i = 1}^{n}[1,x_{i}] = \sum_{i = 1}^{m}[1,y_{i}]$
% implies $\sum_{i = 1}^{n}x_{i} = \sum_{i = 1}^{m}y_{i}$.
% Let $G \colon \mathfrak{T}(\Stc) \to \mathfrak{T}(\Stc)$ be a unique
% traced symmetric monoidal functor
% that makes the following diagram commute:
% \begin{equation*}
%   \xymatrix{
%     \mathfrak{T}(\Stc) \ar[r]^{G} & \mathbf{Mat}(\mathbb{R}_{\geq 0}) \\
%     \Stc \ar[u] \ar[ru] &
%   }
% \end{equation*}
% Since $G$ preserves traces, if
% \begin{math}
%   \sum_{i = 1}^{n}[1,x_{i}] = \sum_{i = 1}^{m}[1,y_{i}]
% \end{math}
% in $\mathfrak{T}(\Stc)$, then
% \begin{math}
%   \sum_{i = 1}^{n} x_{i} = \sum_{i = 1}^{m} y_{i}
% \end{math}
% in $\mathbf{Mat}(\mathbb{R}_{\geq 0})$.
% Now, it is easy to see that $\mathfrak{T}(\Stc)$
% is isomorphic to $\mathbf{Mat}(\mathbb{R}_{\geq 0})$.

\subsection{What about...}

\paragraph{Posets}

Let $\mathbf{pPos}_{\mathrm{fin}}$ be the category of finite
pointed posets and bottom-preserving monotone functions.
To find concrete description of $\mathfrak{T}(\mathbf{pPos}_{\mathrm{fin}})$ is more
difficult than above examples because we can not always
decompose a poset by finite sum of the unit.

\paragraph{Sets}

Let $\mathbf{Set}_{\mathrm{fin}}$ be the category of finite sets and
functions. Since $\fSet$ does not have a zero object, we can not
apply Theorem~\ref{thm:biproduct}. Still, we can show that
$\mathfrak{T}(\fSet)$ is a $\mathbf{CMon}$-enriched symmetric
monoidal category. (See Theorem~\ref{thm:set-like}.) Is it possible to
clarify $\mathfrak{T}(\fSet)$ via $\mathfrak{T}(\fPfn)$ and the
embedding $J \colon \mathbf{Set}_{\mathrm{fin}} \to \fPfn$? At
least, it is not straightforward to do this because
$\mathfrak{T}(J) \colon \mathfrak{T}(\mathbf{Set}_{\mathrm{fin}}) \to
\mathfrak{T}(\fPfn)$ is not faithful. In fact, if a function
$f \colon X \to X$ does not have any fixed point, then
\begin{align*}
  \mathfrak{T}(J)([X,f] \colon 1 \to 1)
  &= \mathfrak{T}(J)(\Tr{X}_{1,1} ([1,f])) \\
  &= \Tr{X}_{1,1} (\mathfrak{T}(J)([1,f])) \\
  &= \Tr{X}_{1,1} ([1,f]) \\
  &= [\emptyset,o_{1,1}].
\end{align*}
When $X$ is not empty, $[X,f] \neq [0,o_{X,X}]$ in $\mathfrak{T}(\mathbf{Set}_{\mathrm{fin}})$.


\section{Another Condition for $\mathbf{CMon}$-enrichment}


\begin{theorem}\label{thm:set-like}
  If a distributive symmetric monoidal category
  $(\mathcal{C},I,\otimes)$ satisfies
  \begin{equation}\label{assumption}\tag{\#}
    \forall X \in \mathcal{C}.\; (X \not\cong 0 \implies
    \exists x \colon I \to X.\;
    \exists x' \colon X \to I.\; x' \circ x  = \Id_{I}),
  \end{equation}
  then $\mathfrak{T}(\mathcal{C})$ is a $\mathbf{CMon}$-enriched
  symmetric monoidal category.
\end{theorem}
\begin{proof}
  We only need to check well-definedness of additin of
  $\mathfrak{T}(\mathcal{C})$-morphisms. We consider
  the following three cases.
  \begin{itemize}
  \item $f = (X \otimes h) \fatsemi p$ and
    $f' = p \fatsemi (Y \otimes h)$ for some $h \colon U \to U'$ and
    $p \colon X \otimes U' \to Y \otimes U$.
  \item $U = W \otimes 0$ and
    $f = (p \otimes 0) \fatsemi (Z \otimes
    \mathrm{sym}_{0,0}) \fatsemi (q \otimes 0)$ and
    $f' = p \fatsemi q$ for some
    $p \colon X \otimes W \to Z \otimes 0$ and
    $q \colon Z \otimes 0 \to Y \otimes W$.
  \item $U = U_{0} \otimes U_{1}$ and
    $f = (p \otimes U_{1}) \fatsemi (Z \otimes
    \mathrm{sym}_{U_{1},U_{1}}) \fatsemi (q \otimes U_{1})$ and
    $f' = p \fatsemi q$ for some $U_{1} \not\cong 0$ and
    $p \colon X \otimes U_{0} \to Z \otimes U_{1}$ and
    $q \colon Z \otimes U_{1} \to Y \otimes U_{0}$.
  \end{itemize}
  The first case can be proved in the same way. The second case
  reduces to the first case because we can derive $f \sim f'$ using
  naturality like the proof of Proposition~\ref{prop:zero}. It remains
  to check the third case. By \eqref{assumption}, there are
  $x \colon I \to U_{1}$ and $x' \colon U_{1} \to I$ such that
  $x \fatsemi x' = \Id_{I}$. Then
  \begin{align*}
    &X \otimes \hide{((U_{0} \otimes U_{1}) \oplus V)}
      \xrightarrow{f \oplus g} Y \otimes \hide{(U_{0} \otimes U_{1} + V)} \\
    &\sim
      X \otimes \hide{((U_{0} \otimes U_{1}) \oplus (V \otimes U_{1}))}
      \xrightarrow{f \oplus (g \otimes (x' \fatsemi x))}
      Y \otimes \hide{((U_{0} \otimes U_{1}) \oplus (V \otimes U_{1}))} \\
    &\sim
      X \otimes \hide{(U_{0} \oplus V) \otimes U_{1}}
      \xrightarrow{X \otimes \mathrm{dst}^{-1}}
      X \otimes ((U_{0} \otimes U_{1}) \oplus (V \otimes U_{1}))
      \xrightarrow{f \oplus (g \otimes (x' \fatsemi x))} \\
    &\hspace{13em}
      Y \otimes ((U_{0} \otimes U_{1}) \oplus (V \otimes U_{1}))
      \xrightarrow{Y \otimes \mathrm{dst}}
      Y \otimes \hide{(U_{0} \oplus V) \otimes U_{1}} \\
    &= X \otimes \hide{(U_{0} \oplus V) \otimes U_{1}}
      \xrightarrow{r \otimes U_{1}}
      (Z \oplus (Y \otimes V)) \otimes U_{1} \otimes U_{1}
      \xrightarrow{(Z \oplus (Y \otimes V)) \otimes \mathrm{sym}} \\
    &\hspace{15em}
      (Z \oplus (Y \otimes V)) \otimes U_{1} \otimes U_{1}
      \xrightarrow{s \otimes U_{1}}
      Y \otimes \hide{(U_{0} \oplus V) \otimes U_{1}}
  \end{align*}
  where
  \begin{align*}
    r &= X \otimes (U_{0} \oplus V)
        \xrightarrow{\mathrm{dst}^{-1}}
        (X \otimes U_{0}) \oplus (X \otimes V)
        \xrightarrow{p \oplus (g \otimes x)} \\
      & \hspace{14em}
        (Z \otimes U_{1}) \oplus (Y \otimes V \otimes U_{1})
        \xrightarrow{\mathrm{dst}}
        (Z \oplus (Y \otimes V)) \otimes U_{1}, \\
    s &= (Z \oplus (Y \otimes V)) \otimes U_{1}
        \xrightarrow{\mathrm{dst}^{-1}}
        (Z \otimes U_{1}) \oplus (Y \otimes V \otimes U_{1})
        \xrightarrow{q \oplus (Y \otimes V \otimes x')} \\
      &\hspace{20em}
        (Y \otimes U_{0}) \oplus (Y \otimes V)
        \xrightarrow{\mathrm{dst}}
        Y \otimes (U_{0} \oplus V).
  \end{align*}
  Because $r \fatsemi s = f' \oplus g \colon X \otimes (U_{0} \oplus V)
  \to Y \otimes (U_{0} \oplus V)$, we obtain
  \begin{equation*}
    X \otimes \hide{((U_{0} \otimes U_{1}) \oplus V)}
    \xrightarrow{f \oplus g}
    Y \otimes \hide{((U_{0} \otimes U_{1}) \oplus V)}
    \sim
    X \otimes \hide{(U_{0} \oplus V)}
    \xrightarrow{f' \oplus g}
    Y \otimes \hide{(U_{0} \oplus V)}.
  \end{equation*}
  We can similarly check that $[U,f] + (-)$ is compatible with
  $\sim$.
\end{proof}

In general, \eqref{assumption} does not imply existence of
coproducts in $\mathfrak{T}(\mathcal{C})$. For example,
there is no cotupling of $\mathfrak{T}(\fSet)$-morphisms
$\Id_{1} \colon 1 \to 1$ and $[0,o_{1,1}] \colon 1 \to 1$.
In fact, if $[U,f] \colon 2 \to 1$ satisfies
$[1,\iota'] \fatsemi [U,f] = [0,o_{1,1}]$, then
$U$ must be the emptyset. Hence,
$[1,\iota] \fatsemi [U,f] = [0,o_{1,1}]$ is not
equal to $\Id_{1}$.

\newpage
\appendix

\section{On Conway Operator}


\begin{proposition}\label{prop:emb0}
  Let $(\mathcal{C},I,\otimes)$ be a symmetric monoidal category
  with an initial object $0$ such that $X \otimes 0 \cong 0$ for
  any $X \in \mathcal{C}$, and let $(\mathcal{D},J,\boxtimes,\Tr)$
  be a traced symmetric monoidal category. If there is a faithful
  symmetric monoidal functor $H \colon \mathcal{C} \to \mathcal{D}$
  and $\mathcal{D}(J,J) = \{\Id_{J}\}$, then $\mathcal{C}$ is a
  preordered category.
\end{proposition}
\begin{proof}
  As usual, we omit canonical isomorphisms.
  For any $\mathcal{C}$-morphism $f \colon X \to Y$, we have
  \begin{equation*}
    Hf = Hf \boxtimes J
       = Hf \boxtimes \Tr{H0}_{J,J} (\Id_{J \boxtimes H0})
       = \Tr{H0}_{HX,HY}(Hf \boxtimes H0).
  \end{equation*}
  Because
  \begin{equation*}
    Hf \boxtimes H0 = H(f \otimes 0)
    = H(X \otimes 0 \xrightarrow{\cong} Y \otimes 0),
  \end{equation*}
  $Hf$ is equal to
  $\Tr{H0}_{HX,HY}(H \otimes 0 \xrightarrow{\cong} Y \otimes 0)$.
  Hence, $Hf = Hg$ for all $f,g \colon X \to Y$. By faithfullness of $H$,
  we obtain $f = g$.
\end{proof}
As a consequence, we see that we can not faithfully extend
$\mathbf{Set}$ to a traced cartesian category in the category of
cartesian categories and cartesian functors.

A symmetric monoidal category $(\mathcal{C},I,\otimes)$
is \emph{well-pointed} when
$\mathcal{C}(I,-) \colon \mathcal{C} \to \mathbf{Set}$ is faithful.
We can replace distributivity $X \otimes 0 \cong 0$ in
Proposition~\ref{prop:emb0} with well-pointedness.
\begin{proposition}\label{prop:emb1}
  Let $(\mathcal{C},I,\otimes)$ be a well-pointed symmetric monoidal
  category with an initial object $0$, and let
  $(\mathcal{D},J,\boxtimes,\Tr)$ be a traced symmetric monoidal
  category. If there is a faithful symmetric monoidal functor
  $H \colon \mathcal{C} \to \mathcal{D}$ with
  $\mathcal{D}(J,J) = \{\Id_{J}\}$, then $\mathcal{C}$ is a
  preordered category.
\end{proposition}
\begin{proof}
  We define a $\mathcal{D}$-morphism $z \colon J \to H0$ to be
  \begin{equation*}
    \Tr{H0}_{J,H0} (J \boxtimes H0 \xrightarrow{\cong}
    H(I \otimes 0) \xrightarrow{\cong} H(0 \otimes 0)
    \xrightarrow{\cong} H0 \boxtimes H0).
  \end{equation*}
  Because $J \xrightarrow{z} H0 \xrightarrow{H\bot_{I}} HI \xrightarrow{\cong} J$
  is equal to the identity,
  $H\bot_{I} \colon 0 \to HI$ is epi. For all
  $f,g \colon X \to Y$ and $x \colon I \to X$,
  we have
  \begin{equation*}
    H\bot_{I} \fatsemi H(x \fatsemi f)
    = H\bot_{Y} = H\bot_{I} \fatsemi H(x \fatsemi g).
  \end{equation*}
  Since $H\bot_{I}$ is epi, $H(x \fatsemi f)$
  is equal to $H(x \fatsemi g)$ for any $x \colon I \to X$. By
  well-pointedness of $\mathcal{C}$ and faithfullness of $H$, we
  obtain $f = g$.
\end{proof}

