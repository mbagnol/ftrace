\section{Building the Free Trace of Monoidal Categories with Coproducts}

\subsection{Preservation of Coproducts}

Generally, it is not easy to see how $\tra{A}$ looks like. Below, we
specify our target to symmetric monoidal categories with coproducts.

\begin{lemma}\label{lem:zero-object}
  If $\catf{A}$ has an initial object $0$ such that the canonical morphism
  $0 \to X \otimes 0$ is an isomorphism for any $X \in \catf{A}$,
  then $0$ is an initial object of $\tra{A}$.
\end{lemma}
\begin{proof}
  For any morphism $(f,U) \colon 0 \to X$ in $\tra{A}$, we have
  \begin{equation*}
    (f,U) = (0 \otimes U \xrightarrow{\cong}  0 \xrightarrow{\cong} X \otimes U,U)
    = (0 \otimes 0 \xrightarrow{\cong}  0 \xrightarrow{\cong} X \otimes 0,0).
  \end{equation*}
  Hence, $0$ is an initial object of $\tra{A}$.
\end{proof}

\begin{proposition}\label{prop:biproduct}
  If $\catf{A}$ has a zero object $0$ and binary coproducts
  \begin{equation*}
    X \xrightarrow{\iota_{X,Y}} X \oplus Y \xleftarrow{\iota'_{X,Y}} Y
  \end{equation*}
  such that the canonoical morphisms
  \begin{equation*}
    (X \otimes Y) \oplus (X \otimes Z) \longrightarrow X \otimes (Y \oplus Z)
  \end{equation*}
  are isomorphism for all $X,Y,Z \in \catf{A}$, then
  $\tra{A}$ has finite coproducts.
\end{proposition}
\begin{proof}
  We have already checked existence of an initial object. Existence of
  binary coproducts follows from that $\Phi_{X,Y,Z} \colon \tra{A}(X \oplus Y,Z) \to
  \tra{A}(X,Z) \times \tra{A}(Y,Z)$ given by
  \begin{equation*}
    \Phi_{X,Y,Z}(f,U) =
    \langle (f,U) \circ \embed{A}(\iota_{X,Y}),\; (f,U) \circ \embed{A}(\iota'_{X,Y}) \rangle
  \end{equation*}
  is bijective. The inverse of $\Phi_{X,Y,Z}$ is given by
  \begin{equation*}
    ((f,U) \colon X \to Y ,(g,V) \colon Y \to Z) \longmapsto (h,U \oplus V)
  \end{equation*}
  where $h \colon (X \oplus Y) \otimes (U \oplus V) \to Z \otimes (U \oplus V)$
  is a unique morphism such that
  \begin{align*}
    h \circ (\iota_{X,Y} \otimes \iota_{U,V})
    &= (Z \otimes \iota_{U,V}) \circ f, \\
    h \circ (\iota_{X,Y} \otimes \iota'_{U,V})
    &= X \otimes V \longrightarrow 0 \longrightarrow Z \otimes (U \otimes V), \\
    h \circ (\iota'_{X,Y} \otimes \iota_{U,V})
    &= Y \otimes U \longrightarrow 0 \longrightarrow Z \otimes (U \otimes V), \\
    h \circ (\iota'_{X,Y} \otimes \iota'_{U,V})
    &= (Z \otimes \iota'_{U,V}) \circ g.
  \end{align*}
  Checking bijecetivity of $\Phi_{X,Y,Z}$ is tedious, but not difficult. For a proof,
  see Appendix....
\end{proof}

\begin{corollary}\label{cor:biproduct}
  Under the assumption in Proposition~\ref{prop:biproduct}.
  \begin{itemize}
  \item The canonical functor $\embed{A} \colon \catf{A} \to \tra{A}$ preserves
    finite coproducts.
  \item $\tra{A}$ has finite \emph{biproducts}, namely, each hom-set has the structure
    of a commutative monoid, and composition is bilinear.
  \end{itemize}
\end{corollary}
\begin{proof}
  It is straightfoward to see preservation of coproducts from the definition
  of coproducts in $\tra{A}$. For finite biproducts, we only describe the
  structure of a commutative monoid on $\tra{A}(X,Y)$. The unit is the zero
  morphism $X \longrightarrow 0 \longrightarrow Y$. Addition of $(f,U),(g,V)
  \colon X \to Y$ is given by
  \begin{equation*}
    (X \otimes (U \oplus V) \xrightarrow{\cong}
    (X \otimes U) \oplus (X \otimes V) \xrightarrow{f \oplus g}
    (Y \otimes U) \oplus (Y \otimes V) \xrightarrow{\cong}
    Y \otimes (U \oplus V), U \oplus V 
    )
  \end{equation*}
  For a proof, see...
\end{proof}

\section{Concrete descriptions}

\subsection{Partial Functions}

Let $(\fPfn,1,\times)$ be the symmetric monoidal category of finite
sets and partial functions. This category has finite coproducts given
by disjoint sum and is distributive. Since every object
$X \in \fPfn$ is isomorphic to a finite coproduct of the monoidal
unit $1 \in \fPfn$, we obtain
\begin{equation*}
  \mathfrak{T}(\fPfn)(X,Y) \cong
  \mathfrak{T}(\fPfn)(1,1)^{X \times Y}
  \cong \left\{\overbrace{[1,\Id_{1}] + \cdots + [1,\Id_{1}]}^{n}
    \mid n \in \mathbb{N}\right\}^{X \times Y}.
\end{equation*}
By universality of $\mathfrak{T}(\fPfn)$, we obtain
the following commutative diagram:
\begin{equation*}
  \xymatrix{
    \mathfrak{T}(\fPfn) \ar[r]^{F} &
    \mathbf{Mat}(\mathbb{N}) \ar[r]^{G} &
    \mathfrak{T}(\fPfn) \\
    & \fPfn \ar[ru]_{H} \ar[u]_{J} \ar[lu]^{H} &
  }
\end{equation*}
where $\mathbf{Mat}(\mathbb{N})$ is the category of finite numbers
and matrices over the semiring $\mathbb{N}$, and
$G \colon \mathbf{Mat}(\mathbb{N}) \to \mathfrak{T}(\fPfn)$
is a symmetric monoidal functor given by
\begin{equation*}
  G((a_{i,j}) \colon n \to m)
  = ([1,a_{i,j}]) \colon n \to m.
\end{equation*}
Because $GF = \Id$, $F$ is faithful, and
because $J$ and $H$ are bijective on objects,
$F$ is bijective on objects.
It is easy to see that any $\mathbf{Mat}(\mathbb{N})$-morphism
is obtained by finite sum of images of $\fPfn$-morphisms.
Since $F$ is traced, $F$ is epi. Hence, $F$ is an isomorphism.


\subsection{Compact Closed Categories}

Because any symmetric monoidal functor from a compact closed category
to a traced symmetric monoidal category is traced, any compact closed
category $\mathcal{C}$ is isomorphic to
$\mathfrak{T}(\mathcal{C})$. For a few concrete compact closed
categories, we can check this using Theorem~\ref{thm:biproduct}. Let
$(\fRel,1,\times)$ be the symmetric monoidal category of finite sets
and relations. This category is traced and has finite biproducts.
Since $H \colon \fRel \to \mathfrak{T}(\fRel)$ preserves finite
biproducts, we obtain the following commutative diagram:
\begin{equation*}
  \xymatrix{
    \mathfrak{T}(\fRel) \ar[r]^{F} &
    \fRel \ar[r]^{G} &
    \mathfrak{T}(\fRel) \\
    & \fRel \ar[ru] \ar[u]_{\Id} \ar[lu] &
  }
\end{equation*}
Since $HF = \Id$, we see that $F$ is full and faithful and
bijective on objects. Hence, $F$ is an isomorphism. In the same way,
we can show that the category $\mathbf{fdVect}$ of finite
demensional vector spaces and linear functions is isomorphic to
$\mathfrak{T}(\mathbf{fdVect})$.

\subsection{Substochastic Matrix}

An $n$-by-$m$ matrix $F=(F_{i,j})$ over $\mathbb{R}_{\geq 0}$ is
\emph{substochastic} when $\sum_{i=1}^{n}F_{i,j} \leq 1$. We define
a category $\Stc$ by: objects are non-negative integers, and
morphisms from $n$ to $m$ are substochastic matrices.
This category inherits symmetric monoidal structure
and coproducts from the category of non-negative integers
and matrices over $\mathbb{R}$. Since $0$ is a zero object
of $\Stc$, by applying Theorem~\ref{thm:biproduct}, we get
\begin{equation*}
  \mathfrak{T}(\Stc)(X,Y) \cong
  \mathfrak{T}(\Stc)(1,1)^{X \times Y}
  \cong \left\{\overbrace{[1,x_{1}] + \cdots + [1,x_{n}]}^{n}
    \mid 0 \leq x_{i} \leq 1 \right\}^{X \times Y}.
\end{equation*}
For $x,y \in [0,1]$ such that $x+y \leq 1$, we have
\begin{align*}
  [1,x + y]
  &= \left[1,
    \begin{pmatrix}
      1 & 1 \\
    \end{pmatrix}\right]
  \fatsemi
  \left[1,
  \begin{pmatrix}
    x \\
    y \\
  \end{pmatrix}
  \right] \\
  &= \left[1,
    \begin{pmatrix}
      1 & 1 \\
    \end{pmatrix}\right]
  \fatsemi
  \left(
  \left[1,
  \begin{pmatrix}
    x \\
    0 \\
  \end{pmatrix}
  \right] +
  \left[1,
  \begin{pmatrix}
    0 \\
    y \\
  \end{pmatrix}
  \right]\right)\\
  &= [1,x] + [1,y].
\end{align*}
% Hence, for any $x_{1},\ldots,x_{n} \in [0,1]$,
% there is a non-negative integer $m$ and $y \in [0,1]$
% such that
% \begin{equation*}
%   [1,x_{1}] + \cdots + [1,x_{n}]
%   = \overbrace{[1,1] + \cdots + [1,1]}^{m} + [1,y].
% \end{equation*}
Hence, for any $x_{1},\ldots,x_{n} \in [0,1]$
and for any $y_{1},\ldots,y_{m} \in [0,1]$,
if $\sum_{i = 1}^{n}x_{i} = \sum_{i = 1}^{m}y_{i}$,
then
\begin{equation*}
  \sum_{i = 1}^{n}[1,x_{i}] = \sum_{i = 1}^{m}[1,y_{i}].
\end{equation*}
As a consequence, we obtain the following
commutative diagram
\begin{equation*}
  \xymatrix{
    \mathfrak{T}(\Stc) \ar[r]^{F} &
    \mathbf{Mat}(\mathbb{R}_{\geq 0}) \ar[r]^{G} & \mathfrak{T}(\Stc) \\
    & \Stc \ar[lu]^{H} \ar[u]_{J} \ar[ru]_{H} &
  }
\end{equation*}
where $G \colon \mathbf{Mat}(\mathbb{R}_{\geq 0}) \to \mathfrak{T}(\Stc)$
is a symmetric monoidal functor given by
\begin{equation*}
  G((a_{i,j})_{i \in n, j \in m} \colon n \to m) =
  ([1,a_{i,j}])_{i \in n, j \in m}.
\end{equation*}
Because $H$ and $J$ are bijective on objects,
$F$ is bijective on objects. Faithfullness of $F$
follows from $GF=\Id$. Furthremore,
since any $\mathbf{Mat}(\mathbb{R}_{\geq 0})$-morphism
is obtained by finite sum of images of $\Stc$-morphisms,
$F$ is full. Hence, $F$ is an isomorphism.



% We show the other direction:
% $\sum_{i = 1}^{n}[1,x_{i}] = \sum_{i = 1}^{m}[1,y_{i}]$
% implies $\sum_{i = 1}^{n}x_{i} = \sum_{i = 1}^{m}y_{i}$.
% Let $G \colon \mathfrak{T}(\Stc) \to \mathfrak{T}(\Stc)$ be a unique
% traced symmetric monoidal functor
% that makes the following diagram commute:
% \begin{equation*}
%   \xymatrix{
%     \mathfrak{T}(\Stc) \ar[r]^{G} & \mathbf{Mat}(\mathbb{R}_{\geq 0}) \\
%     \Stc \ar[u] \ar[ru] &
%   }
% \end{equation*}
% Since $G$ preserves traces, if
% \begin{math}
%   \sum_{i = 1}^{n}[1,x_{i}] = \sum_{i = 1}^{m}[1,y_{i}]
% \end{math}
% in $\mathfrak{T}(\Stc)$, then
% \begin{math}
%   \sum_{i = 1}^{n} x_{i} = \sum_{i = 1}^{m} y_{i}
% \end{math}
% in $\mathbf{Mat}(\mathbb{R}_{\geq 0})$.
% Now, it is easy to see that $\mathfrak{T}(\Stc)$
% is isomorphic to $\mathbf{Mat}(\mathbb{R}_{\geq 0})$.

\subsection{What about...}

\paragraph{Posets}

Let $\mathbf{pPos}_{\mathrm{fin}}$ be the category of finite
pointed posets and bottom-preserving monotone functions.
To find concrete description of $\mathfrak{T}(\mathbf{pPos}_{\mathrm{fin}})$ is more
difficult than above examples because we can not always
decompose a poset by finite sum of the unit.

\paragraph{Sets}

Let $\mathbf{Set}_{\mathrm{fin}}$ be the category of finite sets and
functions. Since $\fSet$ does not have a zero object, we can not
apply Theorem~\ref{thm:biproduct}. Still, we can show that
$\mathfrak{T}(\fSet)$ is a $\mathbf{CMon}$-enriched symmetric
monoidal category. (See Theorem~\ref{thm:set-like}.) Is it possible to
clarify $\mathfrak{T}(\fSet)$ via $\mathfrak{T}(\fPfn)$ and the
embedding $J \colon \mathbf{Set}_{\mathrm{fin}} \to \fPfn$? At
least, it is not straightforward to do this because
$\mathfrak{T}(J) \colon \mathfrak{T}(\mathbf{Set}_{\mathrm{fin}}) \to
\mathfrak{T}(\fPfn)$ is not faithful. In fact, if a function
$f \colon X \to X$ does not have any fixed point, then
\begin{align*}
  \mathfrak{T}(J)([X,f] \colon 1 \to 1)
  &= \mathfrak{T}(J)(\Tr{X}_{1,1} ([1,f])) \\
  &= \Tr{X}_{1,1} (\mathfrak{T}(J)([1,f])) \\
  &= \Tr{X}_{1,1} ([1,f]) \\
  &= [\emptyset,o_{1,1}].
\end{align*}
When $X$ is not empty, $[X,f] \neq [0,o_{X,X}]$ in $\mathfrak{T}(\mathbf{Set}_{\mathrm{fin}})$.


\section{Another Condition for $\mathbf{CMon}$-enrichment}


\begin{theorem}\label{thm:set-like}
  If a distributive symmetric monoidal category
  $(\mathcal{C},I,\otimes)$ satisfies
  \begin{equation}\label{assumption}\tag{\#}
    \forall X \in \mathcal{C}.\; (X \not\cong 0 \implies
    \exists x \colon I \to X.\;
    \exists x' \colon X \to I.\; x' \circ x  = \Id_{I}),
  \end{equation}
  then $\mathfrak{T}(\mathcal{C})$ is a $\mathbf{CMon}$-enriched
  symmetric monoidal category.
\end{theorem}
\begin{proof}
  We only need to check well-definedness of additin of
  $\mathfrak{T}(\mathcal{C})$-morphisms. We consider
  the following three cases.
  \begin{itemize}
  \item $f = (X \otimes h) \fatsemi p$ and
    $f' = p \fatsemi (Y \otimes h)$ for some $h \colon U \to U'$ and
    $p \colon X \otimes U' \to Y \otimes U$.
  \item $U = W \otimes 0$ and
    $f = (p \otimes 0) \fatsemi (Z \otimes
    \mathrm{sym}_{0,0}) \fatsemi (q \otimes 0)$ and
    $f' = p \fatsemi q$ for some
    $p \colon X \otimes W \to Z \otimes 0$ and
    $q \colon Z \otimes 0 \to Y \otimes W$.
  \item $U = U_{0} \otimes U_{1}$ and
    $f = (p \otimes U_{1}) \fatsemi (Z \otimes
    \mathrm{sym}_{U_{1},U_{1}}) \fatsemi (q \otimes U_{1})$ and
    $f' = p \fatsemi q$ for some $U_{1} \not\cong 0$ and
    $p \colon X \otimes U_{0} \to Z \otimes U_{1}$ and
    $q \colon Z \otimes U_{1} \to Y \otimes U_{0}$.
  \end{itemize}
  The first case can be proved in the same way. The second case
  reduces to the first case because we can derive $f \sim f'$ using
  naturality like the proof of Proposition~\ref{prop:zero}. It remains
  to check the third case. By \eqref{assumption}, there are
  $x \colon I \to U_{1}$ and $x' \colon U_{1} \to I$ such that
  $x \fatsemi x' = \Id_{I}$. Then
  \begin{align*}
    &X \otimes \hide{((U_{0} \otimes U_{1}) \oplus V)}
      \xrightarrow{f \oplus g} Y \otimes \hide{(U_{0} \otimes U_{1} + V)} \\
    &\sim
      X \otimes \hide{((U_{0} \otimes U_{1}) \oplus (V \otimes U_{1}))}
      \xrightarrow{f \oplus (g \otimes (x' \fatsemi x))}
      Y \otimes \hide{((U_{0} \otimes U_{1}) \oplus (V \otimes U_{1}))} \\
    &\sim
      X \otimes \hide{(U_{0} \oplus V) \otimes U_{1}}
      \xrightarrow{X \otimes \mathrm{dst}^{-1}}
      X \otimes ((U_{0} \otimes U_{1}) \oplus (V \otimes U_{1}))
      \xrightarrow{f \oplus (g \otimes (x' \fatsemi x))} \\
    &\hspace{13em}
      Y \otimes ((U_{0} \otimes U_{1}) \oplus (V \otimes U_{1}))
      \xrightarrow{Y \otimes \mathrm{dst}}
      Y \otimes \hide{(U_{0} \oplus V) \otimes U_{1}} \\
    &= X \otimes \hide{(U_{0} \oplus V) \otimes U_{1}}
      \xrightarrow{r \otimes U_{1}}
      (Z \oplus (Y \otimes V)) \otimes U_{1} \otimes U_{1}
      \xrightarrow{(Z \oplus (Y \otimes V)) \otimes \mathrm{sym}} \\
    &\hspace{15em}
      (Z \oplus (Y \otimes V)) \otimes U_{1} \otimes U_{1}
      \xrightarrow{s \otimes U_{1}}
      Y \otimes \hide{(U_{0} \oplus V) \otimes U_{1}}
  \end{align*}
  where
  \begin{align*}
    r &= X \otimes (U_{0} \oplus V)
        \xrightarrow{\mathrm{dst}^{-1}}
        (X \otimes U_{0}) \oplus (X \otimes V)
        \xrightarrow{p \oplus (g \otimes x)} \\
      & \hspace{14em}
        (Z \otimes U_{1}) \oplus (Y \otimes V \otimes U_{1})
        \xrightarrow{\mathrm{dst}}
        (Z \oplus (Y \otimes V)) \otimes U_{1}, \\
    s &= (Z \oplus (Y \otimes V)) \otimes U_{1}
        \xrightarrow{\mathrm{dst}^{-1}}
        (Z \otimes U_{1}) \oplus (Y \otimes V \otimes U_{1})
        \xrightarrow{q \oplus (Y \otimes V \otimes x')} \\
      &\hspace{20em}
        (Y \otimes U_{0}) \oplus (Y \otimes V)
        \xrightarrow{\mathrm{dst}}
        Y \otimes (U_{0} \oplus V).
  \end{align*}
  Because $r \fatsemi s = f' \oplus g \colon X \otimes (U_{0} \oplus V)
  \to Y \otimes (U_{0} \oplus V)$, we obtain
  \begin{equation*}
    X \otimes \hide{((U_{0} \otimes U_{1}) \oplus V)}
    \xrightarrow{f \oplus g}
    Y \otimes \hide{((U_{0} \otimes U_{1}) \oplus V)}
    \sim
    X \otimes \hide{(U_{0} \oplus V)}
    \xrightarrow{f' \oplus g}
    Y \otimes \hide{(U_{0} \oplus V)}.
  \end{equation*}
  We can similarly check that $[U,f] + (-)$ is compatible with
  $\sim$.
\end{proof}

In general, \eqref{assumption} does not imply existence of
coproducts in $\mathfrak{T}(\mathcal{C})$. For example,
there is no cotupling of $\mathfrak{T}(\fSet)$-morphisms
$\Id_{1} \colon 1 \to 1$ and $[0,o_{1,1}] \colon 1 \to 1$.
In fact, if $[U,f] \colon 2 \to 1$ satisfies
$[1,\iota'] \fatsemi [U,f] = [0,o_{1,1}]$, then
$U$ must be the emptyset. Hence,
$[1,\iota] \fatsemi [U,f] = [0,o_{1,1}]$ is not
equal to $\Id_{1}$.

\newpage
\appendix

\section{On Conway Operator}


\begin{proposition}\label{prop:emb0}
  Let $(\mathcal{C},I,\otimes)$ be a symmetric monoidal category
  with an initial object $0$ such that $X \otimes 0 \cong 0$ for
  any $X \in \mathcal{C}$, and let $(\mathcal{D},J,\boxtimes,\Tr)$
  be a traced symmetric monoidal category. If there is a faithful
  symmetric monoidal functor $H \colon \mathcal{C} \to \mathcal{D}$
  and $\mathcal{D}(J,J) = \{\Id_{J}\}$, then $\mathcal{C}$ is a
  preordered category.
\end{proposition}
\begin{proof}
  As usual, we omit canonical isomorphisms.
  For any $\mathcal{C}$-morphism $f \colon X \to Y$, we have
  \begin{equation*}
    Hf = Hf \boxtimes J
       = Hf \boxtimes \Tr{H0}_{J,J} (\Id_{J \boxtimes H0})
       = \Tr{H0}_{HX,HY}(Hf \boxtimes H0).
  \end{equation*}
  Because
  \begin{equation*}
    Hf \boxtimes H0 = H(f \otimes 0)
    = H(X \otimes 0 \xrightarrow{\cong} Y \otimes 0),
  \end{equation*}
  $Hf$ is equal to
  $\Tr{H0}_{HX,HY}(H \otimes 0 \xrightarrow{\cong} Y \otimes 0)$.
  Hence, $Hf = Hg$ for all $f,g \colon X \to Y$. By faithfullness of $H$,
  we obtain $f = g$.
\end{proof}
As a consequence, we see that we can not faithfully extend
$\mathbf{Set}$ to a traced cartesian category in the category of
cartesian categories and cartesian functors.

A symmetric monoidal category $(\mathcal{C},I,\otimes)$
is \emph{well-pointed} when
$\mathcal{C}(I,-) \colon \mathcal{C} \to \mathbf{Set}$ is faithful.
We can replace distributivity $X \otimes 0 \cong 0$ in
Proposition~\ref{prop:emb0} with well-pointedness.
\begin{proposition}\label{prop:emb1}
  Let $(\mathcal{C},I,\otimes)$ be a well-pointed symmetric monoidal
  category with an initial object $0$, and let
  $(\mathcal{D},J,\boxtimes,\Tr)$ be a traced symmetric monoidal
  category. If there is a faithful symmetric monoidal functor
  $H \colon \mathcal{C} \to \mathcal{D}$ with
  $\mathcal{D}(J,J) = \{\Id_{J}\}$, then $\mathcal{C}$ is a
  preordered category.
\end{proposition}
\begin{proof}
  We define a $\mathcal{D}$-morphism $z \colon J \to H0$ to be
  \begin{equation*}
    \Tr{H0}_{J,H0} (J \boxtimes H0 \xrightarrow{\cong}
    H(I \otimes 0) \xrightarrow{\cong} H(0 \otimes 0)
    \xrightarrow{\cong} H0 \boxtimes H0).
  \end{equation*}
  Because $J \xrightarrow{z} H0 \xrightarrow{H\bot_{I}} HI \xrightarrow{\cong} J$
  is equal to the identity,
  $H\bot_{I} \colon 0 \to HI$ is epi. For all
  $f,g \colon X \to Y$ and $x \colon I \to X$,
  we have
  \begin{equation*}
    H\bot_{I} \fatsemi H(x \fatsemi f)
    = H\bot_{Y} = H\bot_{I} \fatsemi H(x \fatsemi g).
  \end{equation*}
  Since $H\bot_{I}$ is epi, $H(x \fatsemi f)$
  is equal to $H(x \fatsemi g)$ for any $x \colon I \to X$. By
  well-pointedness of $\mathcal{C}$ and faithfullness of $H$, we
  obtain $f = g$.
\end{proof}



%%% Local Variables:
%%% mode: latex
%%% TeX-master: "MAIN"
%%% End:
