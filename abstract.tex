The notion of trace in a monoidal category allows
to give a category-theoretic account of the notion of feedback, and instances of
it can be found in linear algebra, topology, knot theory and proof theory. An extension to the 
partially defined case has also more recently been studied.

In this article, an explicit construction of the free trace in a symmetric monoidal category is presented,
defined by considering morphisms equiped with a form of private state space and taking a 
suitable quotient.
Conditions for this construction to yield an embedding of the original category into the free traced
one are discussed and called the embedding problem. 
This relates to the notion of partially traced category as it is already known that a symmetric
monoidal category can be equiped with a partial trace if and only if it can be embedded into a
totally traced one.

Examples are considered, using the free construction to determine the free trace on
some standard symmetric monoidal categories, and showing that some classic example of traces are not free.
The extension of the construction to the braided case appears to be straightforward and is sketched in
the end of the article.
