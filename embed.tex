\begin{itemize}
	\item embedding problem and the free trace
	\item more details on failed solutions, counterexaples for the embeding problem
	\item Hassei's note point of view: IY, IYe etc.
	\item partial results and conjecture
\end{itemize}

% commit test


%This leads us to a reformulation of the \enquote{embedding problem}:

%\begin{theorem}[the embedding problem]
%	A monoidal category $\catf C$ can be embedded into a traced one if and only if the functor
%	$\embt C$ is an embedding (or, more explicitely: $f=g\iff (f,\unitobj)\ye(g,\unitobj)$).
%\end{theorem}

%Also, this is related to partial traces since we know \cite{mss12,bagnol15a} that symmetric
%monoidal category can be equipped with a partial trace if and only if it can be embedded into
%a traced category.
%By the way, we can see that the partial trace obtained that way the \enquote{smallest}
%possible (in the sense: the least defined).

%\begin{proposition}
%	If we have two embeddings $E_1,E_2$ of $\catf C$ into traced categories $\catf D_1,\catf D_2$,
%	such that there is a traced functor $G$ making the following diagram commute
%	\begin{center}
%	\begin{tikzpicture}[baseline=-5ex]
%		\node (c) at (0,0) {$\catf C$};
%		\node (tc) at (2,0) {$\catf D_1$};
%		\node (d) at (2,-1.5) {$\catf D_2$};
%		\draw[->] (c) to node [above] {$\funcf E_1$} (tc);
%		\draw[->] (c) to node [below left] {$\funcf E_2$} (d);
%		\draw[->] (tc) to node [right] {$\funcf G$} (d);
%	\end{tikzpicture}
%	\end{center}
%	then the associated partial traces $\Tr{}_1,\Tr{}_2$ on $\catf C$ are related as follows:
%	the domain $D$ of $\Tr{}_1$ is contained in that of $\Tr{}_2$ and the two traces agree on $D$.
%\end{proposition}

%Indeed, the universal property of the previous section ensures that the domain of the partial trace
%induced by $\embt C$ (when it is an embedding) is the smallest possible for a partial trace.

%\subsection*{Embedding condition}

%What we still need to understand is under which conditions on $\catf C$ the functore $\embt C$ is an
%embedding. Ideally we are looking for a condition that is really formulated in the language of
%$\catf C$ and synthetic. We saw that several attempts \enquote{Scott's condition}, 
%\enquote{Selinger's condition} turned out to be necessary but not sufficient.

%For this specific question, I think it will be easier to work in $\state C$, where equality is plain
%equality. This becomes a problem of understanding what can happen when we have a chain of 
%$\eq,\yr{},\yl{}$.

%It is fair to say that we do not really know for now what would a sufficient condition be like.
%So to get started, I think we can explore the properties of chains of $\eq,\yr{},\yl{}$ to
%understand better how they rewrite, what properties they always have \etc
%this could help us pinpoint exactly what is missing to get an embedding, or at least make it easier
%to check whether a new proposed condition is sufficient.

%I'll list some basic results. (I have proof sketches for non-conjecture ones)

%\begin{lemma}[$\eq$ before $\yr{}$]
%	Whenever we have $f\yr{}g\eq g'$ ($f,g,g'$ are in $\state C$) then there exists a $f'$ such that
%	$f\eq f'\yr{}g$. Similarly, $g\eq g'\yl{}f$ implies $g\yl{}f'\eq f$.
%\end{lemma}

%This means that a chain of equivalences $f\quo g$ can always be rewritten as a sequence of blocks of the
%form $h\yl{}^*g\eq g'\yr{}^* f$, where $\yl{}^*$ and $\yr{}^*$ are the transitive closures of 
%$\yl{}$ and $\yr{}$.

%\begin{conjecture}
%	...
%\end{conjecture}


%\subsection*{Naohiko's counterexample}

%Something to discuss when I come to Kyoto: in his original note, Naohiko found a counterexample to
%the sufficiency of Selinger condition, which was attributed to the condition not taking sliding into
%account.

%Yet as we discussed there the IYS condition restricted to sliding only isomorphism is not stronger than
%the IY condition.

%There is an alternative way to look at this, based on the way one gets full sliding from sliding only
%symmetries, where the counterexample appears as rather a failure of \emph{confluence}. I'll just
%include handwritten notes for now as graphical calculations take ages to type, and non-graphical
%are hard to read:







